\chapter{Planificación y Presupuesto}\label{chap:planificacion}
En este capítulo se presentará una visión general del desarrollo y la logística del proyecto. En la sección \sectionref{planificacion} se presentará la planificación del proyecto incluyendo cronograma y la distribución de tareas. Además, se describirá el presupuesto y costes del proyecto (\sectionref{presupuesto}) 

\section{Planificación}\label{sec:planificacion}

Dadas las características del proyecto se ha decidido elegir una metodología iterativa basada en el modelo en espiral de Bohem \cite{ModeloEspiral}. Este modelo es adecuado para proyectos de gran envergadura y complejidad, ya que divide el proyecto en iteraciones, y en cada una de esas iteraciones se realiza un ciclo completo de desarrollo, desde la planificación hasta la entrega del producto en forma de prototipo. Tras completar cada iteración se actualizan las iteraciones siguientes según los resultados obtenidos. Esto permite una mayor flexibilidad y adaptación a los cambios que puedan surgir durante el desarrollo del proyecto.

Cada ciclo de desarrollo se divide en cuatro fases:

\begin{enumerate}
    \item \textbf{Planificación}: Se obtienen los requisitos de usuario y se evaluan para obtener los objetivos de esa iteración.
    \item \textbf{Análisis}: Se identifican y evaluan las alternativas que pueden satisfacer los objetivos fijados.
    \item \textbf{Desarrollo y Pruebas}: Se diseña, desarrolla y prueba la arquitectura propuesta.
    \item \textbf{Evaluación}: El cliente evalua el sistema aportado \textit{feedback}, que se usará como información valiosa en las siguientes iteraciones.
\end{enumerate}



\section{Presupuesto}\label{sec:presupuesto}

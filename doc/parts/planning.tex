\chapter{Planificación y presupuesto}\label{chap:planificacion}
En este capítulo se presentará una visión general del desarrollo y la logística del proyecto. En la sección \sectionref{planificacion} se presentará la planificación del proyecto incluyendo cronograma y la distribución de tareas. Además, se describirá el presupuesto y costes del proyecto (\sectionref{presupuesto}) 

\section{Planificación}\label{sec:planificacion}

\subsection{Metodología}\label{sec:metodologia}
Dadas las características del proyecto, se ha decidido utilizar una metodología en cascada \cite{ModeloCascada} para la planificación y desarrollo del proyecto. Esta metodología se adapta correctamente a un proyecto de una envergadura mediana, ya que permite centrarse en cada una de las fases del proyecto de forma secuencial, y permite una mayor claridad en la planificación y desarrollo del proyecto.

El proyecto se dividirá en 4 grandes etapas: 

\begin{enumerate}
    \item \textbf{Planificación}: Se obtienen los requisitos de usuario y se evaluan para obtener los objetivos de esa iteración.
    \item \textbf{Análisis}: Se identifican y evaluan las alternativas que pueden satisfacer los objetivos fijados.
    \item \textbf{Desarrollo y Pruebas}: Se diseña, desarrolla y prueba la arquitectura propuesta.
    \item \textbf{Evaluación}: El cliente evalua el sistema aportado \textit{feedback}, que se usará como información valiosa en las siguientes iteraciones.
\end{enumerate}

Para cada una de las etapas se realizará el respectivo desarrollo de los siguientes componentes:

\begin{enumerate}
    \item \textbf{Sesión de Depuración}: Se desarrolla el módulo que permite depurar código concurrente a través de GDB/RR, con cada una de las funcionalidades necesarias. 
    \item \textbf{Depuración en Contenedores}: Se desarrolla la capacidad de levantar contenedores Docker para ejecutar las sesiones de depuración.
    \item \textbf{Proxy}: Se desarrolla el proxy que permite la redirección del cliente a las sesiones de depuración, así como la gestión de sesiones de depuración.
    \item \textbf{Comunicaciones}: Se desarrolla la comunicación entre el cliente y el servidor a través de WebSocket.
    \item \textbf{Lado Cliente}: Se desarrolla la interfaz web que permite al usuario interactuar con el sistema.
\end{enumerate} 

\subsection{Estimación de tiempo y cronograma}\label{sec:cronograma}
La estimación de tiempo se ha realizado diseñando un diagrama de Gantt (\figref{gantt}) que permite visualizar la planificación del proyecto. Este diagrama muestra las diferentes fases del proyecto, así como el desarrollo de los distintos componentes para cada una de ellas. Además, se han establecido fechas de inicio y fin para cada tarea, lo que permite tener una visión clara del tiempo necesario para completar el proyecto.

\begin{landscape}
    \thispagestyle{empty}
    \begin{figure}
      \ffigbox[\FBwidth]
        {%
          \caption{Diagram de Gantt del proyecto.}
          \label{fig:gantt}
        }%
        {
          \begin{ganttchart}[
            hgrid,
            % vgrid,
            time slot format=isodate,
            time slot unit=day,
            x unit=.075cm,
            y unit chart=.481cm,
            y unit title=.7cm,
            title label font=\footnotesize,
            group label font=\bf\scriptsize,
            bar label font=\scriptsize,
            milestone label font=\it\scriptsize,
            expand chart=\textwidth
          ]{2024-08-08}{2025-06-18}
  
            \gantttitlecalendar{year, month=shortname} \\
  
            \ganttgroup{I. Planificación}{2024-08-10}{2024-09-24} \\
            \ganttbar{Sesion de depuración}{2024-08-10}{2024-08-20}\\
            \ganttlinkedbar{Depuración en contenedores}{2024-08-21}{2024-09-03}\\
            \ganttlinkedbar{Proxy}{2024-09-04}{2024-09-09}\\
            \ganttlinkedbar{Comunicación}{2024-09-10}{2024-09-17}\\
            \ganttlinkedbar{Lado cliente}{2024-09-18}{2024-09-24}\\

            \ganttgroup{II. Análisis}{2024-09-25}{2024-12-22} \\
            \ganttbar{Sesion de depuración}{2024-09-25}{2024-10-12}\\
            \ganttlinkedbar{Depuración en contenedores}{2024-10-13}{2024-10-26}\\
            \ganttlinkedbar{Proxy}{2024-10-27}{2024-11-11}\\
            \ganttlinkedbar{Comunicación}{2024-11-12}{2024-11-30}\\
            \ganttlinkedbar{Lado cliente}{2024-12-01}{2024-12-22}\\

            \ganttgroup{III. Desarrollo y Pruebas}{2024-12-23}{2025-05-03} \\
            \ganttbar{Sesion de depuración}{2024-12-23}{2025-02-10}\\
            \ganttlinkedbar{Depuración en contenedores}{2025-02-11}{2025-03-02}\\
            \ganttlinkedbar{Proxy}{2025-03-03}{2025-03-29}\\
            \ganttlinkedbar{Comunicación}{2025-03-30}{2025-04-20}\\
            \ganttlinkedbar{Lado cliente}{2025-04-21}{2025-05-03}\\
  
            \ganttgroup{IV. Evaluación}{2025-05-04}{2025-06-13} \\
            \ganttbar{Sesion de depuración}{2025-05-04}{2025-05-12}\\
            \ganttlinkedbar{Depuración en contenedores}{2025-05-13}{2025-05-19}\\
            \ganttlinkedbar{Proxy}{2025-05-20}{2025-05-24}\\
            \ganttlinkedbar{Comunicación}{2025-05-25}{2025-06-01}\\
            \ganttlinkedbar{Lado cliente}{2025-06-02}{2025-06-13}\\
  
            \ganttgroup{Memoria}{2025-01-15}{2025-06-16} \\
  
  
            % link iterations
            \ganttlink{elem0}{elem6}
            \ganttlink{elem6}{elem12}
            \ganttlink{elem12}{elem18}
          \end{ganttchart}
        }
    \end{figure}
  \end{landscape}


\section{Presupuesto}\label{sec:presupuesto}

En esta sección se presentará el presupuesto del proyecto, basado en la estimación de tiempo y planificación presentada en la sección \sectionref{cronograma}. 

\begin{table}[htbp]
    \centering
    \caption{Resumen del presupuesto del proyecto}
    \label{tab:resumen_presupuesto}
    \begin{tabular}{@{}l|l@{}}
    \toprule
    \textbf{Titulo}          & Herramienta Didáctica pra la Programación Concurrente \\ \midrule
    \textbf{Autor}           & Adrián Fernández Galán                                \\ \midrule
    \textbf{Fecha de Inicio} & 10-08-2024                                            \\ \midrule
    \textbf{Fecha Final}     & 16-06-2025                                            \\ \midrule
    \textbf{Duración}        & 10 meses                                              \\ \midrule
    \textbf{Presupuesto}     & 12.702,30\euro                                            \\ \bottomrule
    \end{tabular}
\end{table}

El desglose de los costes se dividirá en dos costes: costes directos y costes indirectos. Estos costes no incluyen impuestos, sino que se incluirán en \subsectionref{oferta_propuesta}

\subsection{Costes directos}\label{subsec:costes_directos}

Los costes directos son aquellos que están directamente relacionados con el desarrollo del proyecto. Estos se dividen en dos subgrupos: costes de personal y costes de equipamiento.
\begin{itemize}
    \item \textbf{Costes de personal}: Estos costes dependerán de los roles, las responsabilidades y el tiempo de desarrollo de cada uno de los miembros
    \item \textbf{Costes de equipamiento}: Estos costes están relacionados con las herramientas usadas durante el desarrollo del proyecto
\end{itemize}

\subsubsection{Costes de personal}
Durante el desarrollo del proyecto han sido necesarios 4 roles diferentes:
\begin{itemize}
    \item \textbf{Jefe de Proyecto}: Su tarea principal es planificar y coordinar el proyecto.
    \item \textbf{Analista}: Se encarga de analizar los requisitos de usuario y diseñar la arquitectura del proyecto.
    \item \textbf{Desarrollador}: Su tarea es implementar las funcionalidades indicadas en los requisitos de software.
    \item \textbf{Evaluador}: Se encarga de diseñar y ejecutar las pruebas del sistema, así como de evaluar el sistema y aportar \textit{feedback} al equipo de desarrollo.
\end{itemize}

En la tabla \tabref{costes_personal} se presenta el desglose de los costes de personal del proyecto. Estos costes se han calculado en base a las horas estimadas de trabajo y el coste por hora de cada uno de los roles. El coste total del personal del proyecto es de 10.650\euro.

\begin{table}[htbp]
    \centering
    \caption{Costes de personal del proyecto}
    \label{tab:costes_personal}
    \begin{tabular}{@{}lccc@{}}
    \toprule
    \textbf{Rol}     & \textbf{Horas} & \textbf{Coste por Hora} & \textbf{Total}   \\ \midrule
    Jefe de Proyecto & 45h            & 60\euro                 & 2.700\euro       \\
    Analista         & 70h            & 35\euro                 & 2.450\euro       \\
    Desarrollador    & 150h           & 32\euro                 & 4.800\euro       \\
    Evaluador        & 35h            & 20\euro                 & 700\euro         \\ \midrule
    \textbf{Total}   & 300h           &                         & \textbf{10.650\euro} \\ \bottomrule
    \end{tabular}
\end{table}

\subsubsection{Costes de equipamiento}
Para el desarrollo del proyecto se han necesitado diferentes herramientas y recursos. \tabref{costes_equipamiento} muestran los costes de equipamiento del proyecto. Estos costes se han calculado en base a la \eqref{eq:amortizacion}. 

\begin{equation}
    C = \frac{c \cdot u \cdot t}{a}
    \label{eq:amortizacion}
\end{equation}


Donde:
\begin{itemize}
    \item $C$: Coste de amortización del recurso
    \item $c$: Coste del recurso
    \item $u$: Uso del recurso (horas)
    \item $t$: Tiempo de amortización del recurso (horas)
    \item $a$: Amortización del recurso (horas)
\end{itemize}

\begin{table}[htbp]
    \begin{adjustbox}{max width=\textwidth}
    \centering
    \caption{Costes de equipamiento del proyecto}
    \label{tab:costes_equipamiento}
    \begin{tabular}{@{}lccccc@{}}
    \toprule
    \textbf{Concepto} & \multicolumn{1}{l}{\textbf{Coste (c)}} & \multicolumn{1}{l}{\textbf{Uso (u)}} & \multicolumn{1}{l}{\textbf{Tiempo usado (t)}} & \multicolumn{1}{l}{\textbf{Amortización (a)}} & \multicolumn{1}{l}{\textbf{Coste Amortizado (C)}} \\ \midrule
    Portátil          & 900,00\euro                                & 45\%                                 & 10 meses                                      & 48 meses                                      & 84,38\euro                                            \\
    Monitor           & 125,95\euro                                & 60\%                                 & 10 meses                                      & 60 meses                                      & 12,60\euro                                            \\
    Teclado           & 50,00\euro                                 & 60\%                                 & 10 meses                                      & 36 meses                                      & 8,33\euro                                             \\
    Silla Oficina     & 120,00\euro                                & 40\%                                 & 10 meses                                      & 60 meses                                      & 8,00\euro                                             \\
    Software          & 0,00\euro                                  & 40\%                                 & 10 meses                                      & 120 meses                                     & 0,00\euro                                             \\ \midrule
    \textbf{Total}    & 1.195,95\euro                              &                                      &                                               &                                               & \textbf{113,30\euro}                                  \\ \bottomrule
    \end{tabular}
    \end{adjustbox}
\end{table}


\subsection{Costes indirectos}\label{subsec:costes_indirectos}
Los costes indirectos son aquellos que ocurren durante el proyecto, pero no pueden ser directamente relacionados con un producto o servicio específico.

Para el consumo de electricidad, se ha revisado la factura de la vivienda durante los 10 meses de desarrollo del proyecto, la cual ha sido de 2,13\euro el día. Teniendo en cuenta que el proyecto ha durado 10 meses, y que se han trabajado 30 horas al mes, el coste total de electricidad es de 2,13\euro \textit{día} $\cdot$ 30 \textit{días} $\cdot$ 10 \textit{meses} = 639\euro. 

En cuanto a los costes de internet, se tiene contratada en la vivienda una tarifa de 50\euro al mes. Y en cuanto al transporte, se ha estimado en torno a 80\euro al mes para los viajes realizados al centro educativo.

El resultado de todos los costes indirectos se puede ver en la tabla \tabref{costes_indirectos}.

\begin{table}[htbp]
    \centering
    \caption{Costes indirectos del proyecto}
    \label{tab:costes_indirectos}
    
    \begin{tabular}{@{}lccc@{}}
    \toprule
    Recurso        & Coste Unitario & Tiempo   & \textbf{Total} \\ \midrule
    Electricidad   & 2,13\euro/día      & 10 meses & 639\euro           \\
    Internet       & 50\euro            & 10 meses & 500\euro           \\
    Transporte     & 80\euro            & 10 meses & 800\euro           \\ \midrule
    \textbf{Total} &                &          & 1.939\euro         \\ \bottomrule
    \end{tabular}
    \end{table}

\subsection{Costes totales}\label{subsec:costes_totales}

Los costes totales del proyecto constan de la suma de los costes directos, tanto de los costes de personal como de los costes de equipamiento, y los costes indirectos. El resultado de la suma de todos los costes se puede ver en la tabla \tabref{costes_totales}.

\begin{table}[htbp]
    \centering
    \caption{Costes totales del proyecto}
    \label{tab:costes_totales}
    \begin{tabular}{@{}lc@{}}
    \toprule
    Personal                & 10.650,00\euro \\
    Equipamiento            & 113,30\euro    \\
    Indirecto               & 1.939,00\euro  \\ \midrule
    \textbf{Total Proyecto} & \textbf{12.702,30\euro} \\ \bottomrule
    \end{tabular}
    \end{table}

Es por esto que el coste total del proyecto es de \textbf{12.702,30\euro}.

\subsection{Oferta propuesta}\label{subsec:oferta_propuesta}

Conociendo el coste total del proyecto, se ha decidido realizar una oferta al cliente. Esta oferta contempla un riesgo del 15\% sobre el coste total y un margen de beneficio del 20\%. Tras esto se le aplicará el IVA correspondiente del 21\%. El resultado de la oferta se puede ver en la tabla \tabref{oferta_propuesta}.


\begin{table}[htbp]
    \centering
    \caption{Oferta propuesta al cliente}
    \label{tab:oferta_propuesta}
    \begin{tabular}{@{}lrcc@{}}
    \toprule
    \textbf{Concepto}  & \multicolumn{1}{l}{\textbf{Incremento}} & \multicolumn{1}{l}{\textbf{Coste Parcial}} & \multicolumn{1}{l}{\textbf{Coste Agregado}} \\ \midrule
    Coste del Proyecto & -                                       & 12.702,30\euro                                 & 12.702,30\euro                                  \\
    Riesgo             & 15\%                                    & 1.905,35\euro                                  & 14.607,65\euro                                  \\
    Beneficio          & 20\%                                    & 2.921,53\euro                                  & 17.529,18\euro                                  \\
    Impuestos          & 21\%                                    & 3.681,13\euro                                  & 21.210,31\euro                                  \\
    \textbf{Total}     & 56\%                                    &                                            & \textbf{21.210,31\euro}                         \\ \bottomrule
    \end{tabular}
    \end{table}

\chapter{Implementación y Despliegue}\label{chap:implementacion_y_despliegue}
En este capítulo se describen los aspectos relacionados con la implementación y el despliegue del sistema. Se detallan las herramientas utilizadas (\subsectionref{uso_librerias_dependencias}), la estructura de ficheros del proyecto (\subsectionref{estructura_ficheros_proyecto}), la imagen docker desarrollada (\subsectionref{construccion_imagen_docker}) y el proceso de despliegue (\sectionref{despliegue}).

\section{Implementación} \label{sec:implementacion}
Tal y como se mencionó en \sectionref{programacion}, el lado cliente será desarrollado con JavaScript, mientras que el lado servidor (proxy + contenedores) se desarrollará en Python. Además el lado servidor se comunicará con un contenedor docker para ejecutar de manera aislada el código desarrollado por el cliente (\subsectionref{sandbox}). Con esto en mente se explicará como se han organizado cada parte del proyecto, cómo se han integrado las librerías y cómo se ha construido la imagen docker y el archivo \texttt{docker-compose.yml} para la orquestación de los contenedores.


El código al completo se encuentra disponible en el repositorio de GitHub del proyecto \href{https://github.com/Adri-Extremix/Trabajo-de-Fin-de-Grado}.

\subsection{Estructura de ficheros del proyecto}\label{subsec:estructura_ficheros_proyecto}

Se ha decidido tener un primer nivel de directorios que separen las distintas partes del proyecto. La estructura se muestra en \figref{1-nivel-dir}.   
\begin{figure}[H]
    \ffigbox[\FBwidth]
    {%
    \caption{Primer nivel de directorios del proyecto}
    \label{fig:1-nivel-dir}
    }
    {
    \begin{tcolorbox}
        \dirtree{%
        .1 /. 
            .2 src/. 
            .2 examples/.
            .2 test/.
            .2 README.md.
            .2 doc/.
            .2 run\_proyect.py.       
        }
    \end{tcolorbox}
    }
\end{figure}
En el directorio raíz del proyecto se encuentran los ficheros del código fuente en el directorio \texttt{src} y el directorio \texttt{examples} contiene ejemplos de códigos que podrían escribir los usuarios. En el directorio \texttt{test} se encuentran los ficheros necesarios para realizar las pruebas de integración y de unidad. El fichero \texttt{README.md} contiene la información necesaria para ejecutar el proyecto y el directorio \texttt{doc} se encuentra todo lo relacionado con la memoria sobre el proyecto.

A partir de aquí exploraremos el directorio del código fuente. En el directorio \texttt{src} se encuentran los ficheros de código fuente del lado cliente, del lado servidor y del contenedor docker. La estructura de este directorio se muestra en \figref{src-dir}.

\begin{figure}[htb]
    \ffigbox[\FBwidth]
    {%
    \caption{Estructura del directorio src}
    \label{fig:src-dir}
    }
    {
    \begin{tcolorbox}
        \dirtree{%
        .1 src/.
            .2 frontend/.
                .3 js/.
                    .4 script.js.
                    .4 code\_editor.js.
                    .4 client\_api.js.
                .3 images/.
                .3 css/.
                .3 index.html.
                .3 package.json.
            .2 proxy/.
                .3 proxy.py.
                .3 requirements.txt.
                .3 Dockerfile.
                .3 docker\-\_compose.yml.
            .2 docker/.
                .3 compiler.py.
                .3 webSocket.py.
                .3 debugger.py.
                .3 Dockerfile.
                .3 requirements.txt.
                .3 docker\-\_compose.yml.
            .2 run\_back\_docker.sh.
        }
    \end{tcolorbox}
    }
\end{figure}
En el directorio \texttt{proxy} se encuentra el fichero \texttt{proxy.py} que contiene la implementación del proxy, el fichero \texttt{requirements.txt} que contiene las dependencias necesarias y los ficheros \texttt{Dockerfile} y \texttt{docker-compose.yml} que contienen la configuración del contenedor proxy.

En el directorio del lado cliente \texttt{frontend} se encuentra el fichero \texttt{index.html} que contiene la estructura de la página web, el fichero \texttt{package.json} que contiene las dependencias del lado cliente, y un directorio para cada tipo de recurso (imágenes, scripts y estilos). 
En el directorio \texttt{docker} se encuentran los ficheros necesarios para construir la imagen docker. En este directorio se encuentran los ficheros \texttt{Dockerfile} y \texttt{docker-compose.yml} que contiene las instrucciones para construir la imagen. También se incluye un fichero \texttt{requirements.txt} que contiene las dependencias necesarias para ejecutar el código del contenedor, además de cada uno de los ficheros \texttt{.py} que contienen la implementación de cada uno de los componentes de cada contenedor.  

Por último, en el directiorio \texttt{src} se encuentra el fichero \texttt{run\_proyect.sh} que permite desplegar la parte correspondiente al lado servidor. 

\subsection{Uso de librerías y dependencias} \label{subsec:uso_librerias_dependencias}

En esta sección se describen las librerías y dependencias utilizadas en el proyecto, así como su propósito y cómo se integran en el sistema. Las dependencias y librerías se engloban en dos grupos: las del lado cliente y las del lado servidor.

\subsubsection{Lado cliente}
Dado que el lado cliente se desarrolla en JavaScript, se ha usado NodeJS como entorno de ejecución y gestor de paquetes. Para la gestión de dependencias se ha utilizado \texttt{npm} y \texttt{package.json} para definir las dependencias necesarias.
Las principales dependencias utilizadas son:
\begin{itemize}
    \item \textbf{Parcel}: Parcel me permite empaquetar el código JavaScript y sus dependencias en un solo archivo, facilitando la carga y ejecución en el navegador.Permitiendo que se puedan servir los ficheros desde el lado servidor a los navegadores.
    \item \textbf{CodeMirror}: CodeMirror es una librería que proporciona un editor de código enriquecido, que permite resaltar la sintaxis y ofrece funcionalidades avanzadas.
    \item \textbf{Socket.io}: Socket.io es una librería que permite la comunicación en tiempo real entre el cliente y el servidor a través de WebSockets. Es una librería fundamental para permitir comunicar el lado cliente con la sesión de depuración que se ejecuta en los diferentes contenedores. 
\end{itemize}

\subsubsection{Lado servidor}
El lado servidor se desarrolla en Python y está conformado por el proxy y los contenedores. Tal y como se ha desarrollado en \subsectionref{sandbox}, el proxy se encarga de gestionar la comunicación entre el cliente y los contenedores, mientras que los contenedores se encargan de ejecutar el código del cliente. 
Para la construcción de los contenedores se ha usado Docker y Docker Compose, que permiten levantar unos sistemas aislados con una serie de características predefinas.
Para la gestión de dependencias se ha utilizado \texttt{pip} y \texttt{requirements.txt} para definir las dependencias necesarias. 
Como principales librerías utizadas:
\begin{itemize}
    \item \textbf{Flask}: Flask es un microframework para Python que permite crear aplicaciones web de manera sencilla y rápida. Flask permite crear endpoints para la comunicación entre cliente, proxy y contenedores, tanto con \textit{API REST}, como con \textit{Web Socket} a través de la extensión \texttt{Flask-SocketIO}.
    \item \textbf{PyGDBMI}: PyGDBMI es una librería que facilita el uso y la comunicación con un subproceso que ejecuta GDB. De esta manera se puede usar GDB para depurar la comunicación a través de un método \texttt{write}, donde se indica qué acción debe de realizar GDB y se obtien como resultado del método un diccionario con la salida. 
\end{itemize}

\subsection{Construcción de imagen Docker y Docker-compose} \label{subsec:construccion_imagen_docker}

Como se ha indicado con anterioridad, los contenedores se construyen a partir de una imagen Docker, que se define en el fichero \texttt{Dockerfile}. Y los diferentes contenedores se orquestan a través de un fichero \texttt{docker-compose.yml}. Tanto el proxy como los contenedores tienen su propio \texttt{Dockerfile} y \texttt{docker-compose.yml}. En esta sección se explicará cómo se construyen las imágenes y cómo se orquestan los contenedores.

\subsubsection{Proxy}

La gestión del proxy es sencilla y tan solo es necesario crear un docker que soporte python, instalar las dependencias, abrir el puerto y ejecutar el proxy. El \texttt{Dockerfile} del proxy se muestra a continuación en la figura \ref{lst:dockerfile-proxy}. 

\begin{lstlisting}[caption={Dockerfile para el proxy}, label={lst:dockerfile-proxy}]
FROM python:3.12

WORKDIR /app

COPY requirements.txt .
COPY proxy.py .
RUN pip install -r requirements.txt

EXPOSE 8080

CMD ["python", "proxy.py"]
\end{lstlisting}

Para comunicar el proxy con el resto de contenedores se ha indicado en el \texttt{docker-compose.yml} el nombre del contenedor y el nombre de la red en la que se encuentra, para facilitar la conexión cuando todos los contenedores se encuentran en la misma máquina.

\begin{lstlisting}[caption={Docker-compose.yml para el contenedor del proxy}, label={lst:docker-compose-proxy}]
    services:
  proxy:
    container_name: proxy  
    build:
      dockerfile: Dockerfile
    ports:
      - "8080:8080" 
    volumes:
      - .:/app
    command: python proxy.py
    environment:
      - ALLOWED_HOSTS=*
    networks:
      - debugger_network

networks:
  debugger_network:
    external: true 
\end{lstlisting}

\subsubsection{Contendores de Depuración}

En cuanto a los contenedores se encargan de las sesiones de depuración se partirá de un único \texttt{Dockerfile}. Este \texttt{Dockerfile} se basará en una imagen de Ubuntu y se instalará todo lo necesario para poder ejecutar python y subprocesos con gdb y rr.

Además se copiarán los archivos fuente relaciondos con las sesiones de depuración y los ficheros estáticos del lado cliente para servirlos.

\begin{lstlisting}[caption={Dockerfile para el contenedor de depuración}, label={lst:dockerfile-debugger}]
FROM ubuntu:latest

RUN apt-get update && apt-get install -y \
    build-essential \
    python3.12 \
    gdb \
    rr \
    # Entre otros

WORKDIR /app

COPY docker/compiler.py docker/debugger.py docker/requirements.txt docker/webSocket.py ./
COPY ../frontend/dist ./frontend/dist

# Configurar Python e instalar dependencias en un entorno virtual
RUN python3.12 -m venv venv && \
    . venv/bin/activate && \
    pip install --upgrade pip && \
    pip install -r requirements.txt

EXPOSE 5000

CMD ["/app/venv/bin/python3", "webSocket.py"]
\end{lstlisting}


Con la imagen preparada solo es necesario indicar cómo se deben de desplegar. Para ello se usará el fichero \texttt{/src/docker/docker-compose.yml}. Este fichero tendrá la configuración de las variables de entorno que se usan en el código, con la información sobre los host y puertos, de la red a la que se tiene que contectar y del mapeo de puertos que tiene que realizar el contenedor. Además se indicarán los recursos que podrán usar cada docker, en este caso se ha decidido que solo puedan usar 4 hilos y 128MB reservados para ese contenedor, con hasta un máximo de 256MB de memoria dinámica. En la figura \ref{lst:docker-compose-debugger} se muestra un fragmento del \texttt{docker-compose.yml} que corresponde con la configuración de un contenedor de depuración llamado \texttt{debugger1}. 

\begin{lstlisting}[caption={Docker-compose.yml para un contenedor de depuración}, label={lst:docker-compose-debugger}]
services:
    debugger1:
        build:
            context: ../ 
            dockerfile: docker/Dockerfile
        environment:
            - PROXY_URL=http://proxy:8080 
            - HOST=debugger1
            - PORT=5000
        ports:
            - "5001:5000"
        networks:
            - debugger_network
        volumes:
            - ./compiler.py:/app/compiler.py
            - ./debugger.py:/app/debugger.py
            - ./webSocket.py:/app/webSocket.py
            - ../frontend/dist:/app/dist  
        deploy:
            resources:
                limits:
                cpus: '4.0'
                memory: 256M
                reservations:
                    memory: 128M
\end{lstlisting}

\section{Despliegue} \label{sec:despliegue}

El despligue del lado puede ser desplagado separando en distintas máquinas el proxy y los contenedores, o bien desplegando todo en la misma máquina. En este caso se ha decidido desplegar el proxy y los contenedores en la misma máquina, pero se puede desplegar de manera separada. Por esta razón las especificaciones técnicas se han separado en los 3 principales componentes aunque el despligue se realice en la misma máquina.

Las especificaciones técnicas recomendadas para el despliegue del sistema son las siguientes:


% Please add the following required packages to your document preamble:
% \usepackage{booktabs}
\begin{table}[htb]
    \begin{adjustbox}{max width=\textwidth}  % permite que la tabla se desborde y quede centrada

    \begin{tabular}{@{}lccc@{}}
    \toprule
    Especificación    & \textbf{Servicio de Sesiones de Depuración}                                                                & \textbf{Máquina Proxy}            & \textbf{Máquina de Lado Cliente}  \\ \midrule
    CPU               & \begin{tabular}[c]{@{}c@{}}4 núcleos $\times$ (número de sesiones)\\ Intel Core i7-13700 or superior\end{tabular} & Intel® CoreTM i3 CPU 6300 & CPU con al menos 2 núcleos \\
    RAM               & \begin{tabular}[c]{@{}c@{}}512 Mb $\times$ (número de sesiones) \\ 4 GB o superior\end{tabular}                   & 2 GB o superior                   & 2GB o superior                    \\
        Almacenamiento    & 2 GB $\times$ (número de sesiones)                                                                                & 2 GB o superior                   & 100 MB                            \\
        Sistema Operativo & Ubuntu 24.04 LTS                                                                                           & Ubuntu 24.04 LTS                  & Cualquiera                        \\ \bottomrule
    \end{tabular}
    \end{adjustbox}

    \end{table}

Todas las máquinas deben de estar conectadas a Internet y tener acceso a los puertos usados en los contenedores. 

El software necesario debe de ser el siguiente:
\begin{itemize}
    \item \textbf{Lado Servidor} (Proxy + Contenedores):
    \begin{itemize}
        \item Docker
        \item Docker Compose
    \end{itemize}
    \item \textbf{Lado Cliente}:
    \begin{itemize}
        \item Navegador web moderno (Chrome, Firefox, Edge, etc.)
        \item NodeJS
        \item npm
    \end{itemize}   
\end{itemize}

Para poder desplegar el proyecto se debe de seguir la guía indicada en \appref{manual}.
En resumen, para realizar un despligue de prueba se debe de ejecutar el script que se encuentra en la raíz del proyecto \texttt{start\_proyect.sh}. Sin embargo, si se quiere hacer un despligue más específico, se puede desplegar solo el lado servidor a través del script \texttt{run\_back\_docker.sh} que se encuentra en el directorio \texttt{src}. Este script ejecutará el proxy y los contenedores de depuración.
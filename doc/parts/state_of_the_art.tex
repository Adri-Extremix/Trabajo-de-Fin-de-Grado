\chapter{Estado del arte}\label{chap:estado-del-arte}
En este capítulo se describirá el estado actual de las tecnologías y herramientas relacionadas con el proyecto. Se explorará la importancia de la concurrencia y las distintas herramientas de \glsdisp{depurar}{depuración} de código que permitan analizar el comportamiento de \gls{programa concurrente} y cómo sus características casan con los objetivos del proyecto (\sectionref{objetivos}).

\section{Concurrencia y su Importancia}\label{sec:concurrencia-importancia}

La concurrencia se define como la capacidad de un sistema para ejecutar múltiples glspl{proceso} de forma simultánea o intercalada, según lo determine el \gls{planificador} del sistema operativo. Gracias a este mecanismo, un programa puede gestionar diversas tareas como la interacción con el usuario, el procesamiento de datos y la comunicación en red sin que el bloqueo de una impida el progreso de las demás, lo que optimiza el rendimiento general y reduce el tiempo de respuesta.

Este enfoque no solo mejora la eficiencia en la ejecución, sino que también incrementa la escalabilidad, especialmente en entornos distribuidos, y contribuye a disminuir el consumo energético mediante una distribución adecuada de las tareas y la reducción de las frecuencias de reloj \cite{EnergyEfficiencyParallelAplications}. Además, la desaceleración en el avance de la Ley de Moore \cite{LeyMoore} ha subrayado la necesidad de explorar otros medios para aumentar el rendimiento y la eficiencia de los sistemas \cite{ParallelProgramming}.

Sin embargo, diversas investigaciones sobre la paralelización en \gls{tiempo de compilación} han evidenciado que extraer de forma automática el \gls{paralelismo} de programas secuenciales resulta un desafio considerable \cite{ParallelProgramming}. Esto implica que, para aprovechar plenamente las ventajas de los \gls{sistemas multicore}, es indispensable que los desarrolladores realicen una reestructuración manual del código para transformarlo en una aplicación concurrente.

En este contexto, formar nuevos profesionales con conocimientos profundos en \gls{concurrencia} se presenta como una estrategia clave para mejorar el rendimiento de las aplicaciones y reducir los costes energéticos, consolidándose como un factor competitivo esencial en el desarrollo de \gls{software} actual y futuro.


\section{Depuradores de Programas Concurrentes}\label{sec:depuradores-programas-concurrentes}

Un \gls{depurador} \cite{WikipediaDebugger} es un programa usado para probar y \gls{depurar} otros programas. Los \glspl{depurador} suelen permitir colocar puntos de interrupción, saltar a través de distintas partes del código e inspeccionar el estado de la memoria, registros de la \gls{CPU} y la pila de llamadas.
Estos \glspl{depurador} suelen ser integrados en los entornos de desarrollo, como \textit{Visual Studio} \cite{DebuggerVisualStudio} o \textit{Eclipse} \cite{DebuggerEclipse}, aunque también existen \glspl{depurador} basados en línea de comandos (\gls{CLI}) como \textit{GDB} \cite{GDB} o \textit{LLDB} \cite{LLDB}. 

.
Sin embargo, los \glspl{depurador} tradicionales no suelen estar diseñados para \glspl {programa concurrente}. En un \gls{programa concurrente} \cite{ParallelismComputerArchitecture} existen múltiples \glspl{proceso}, que pueden ser ejecutados en paralelo o de forma intercalada, que interactuan con los recursos, lo que obliga a que los \glspl{depurador} tengan funcionalidades específicas para soportar este tipo de ejecuciones.

Los \glspl{depurador} de programas concurrentes son capaces de controlar e inspeccionar cada uno de los \glspl{proceso} que existen en el programa, aunque el nivel de granularidadd y funcionalidad dependen de la herramienta. A continuación se explorarán algunas de las herramientas usadas en la actualidad.

\subsection{GDB}{\label{subsec:gdb}}
GDB es un \gls{depurador} de código abierto creado por la \textit{Free Software Foundation} en 1986 \cite{GDB}. Este \gls{depurador} solo está disponible para sistemas \textit{Unix} \cite{UNIX} , aunque simulando este sistema en otros sistemas operativos se puede utilizar \cite{GDBDownload}. Es importante destacar que esta herramienta ocupa en disco únicamente en torno a 12.22 MB. 
GDB una herramienta con la que se interactua a través de su \gls{CLI} o \gls{TUI} y permite controlar la ejecución de programas y analizar su estado en tiempo de ejecución. GDB es capaz de depurar programas escritos en C, C++, D, Fortran, Go, Pascal, Rust y \gls{ensamblador} \cite{GDB}.  

GDB permite colocar puntos de interrupción, inspeccionar la memoria, los \gls{registros} de la CPU y la pila de llamadas e incluso revisar el código \gls{ensamblador} del programa y modificar el valor de las variables en tiempo de ejecución. Además presenta ciertas características avanzadas como la posibilidad de crear \glspl{script} en Python para automatizar tareas \cite{GDBPython}, utilizar \gls{lenguaje máquina} para devolver respuestas procesadas y entendibles por las máquinas \cite{GDB/MI} controlar la ejecución del programa hacia atrás, aunque esta última solo está disponible para programas mono-hilo \cite{GDBReversing}.

Sus puntos fuertes son la multitud de funcionalidades que ofrece para \gls{depurar} y obtener información sobre el estado del programa y posibilidad de automatizar sus \glspl{proceso} a través de la \gls{API} de Python o su \gls{interfaz máquina-humano}, lo que permite crear nuevas herramientas que extiendan las características de este \gls{depurador} y se adapten a las necesidades del usuario. Ejemplos de ello son CLion (\subsectionref{clion}) o herramientas menos cononidas como Blink \cite{Blink}, un \gls{depurador} de entornos mixtos que utiliza la composición para \gls{depurar} programas que utilizan varios códigos.

\begin{lstlisting}[caption={Muestra del depurador GDB}]
(gdb) break 50
Breakpoint 1 at 0x1324: file prueba1.c, line 50.
(gdb) break 60
Breakpoint 2 at 0x1376: file prueba1.c, line 60.
(gdb) run
Starting program: /home/adrian/Documents/Trabajo-de-Fin-de-Grado/examples/prueba1.o 
[New Thread 0x7ffff7a006c0 (LWP 39887)]
[Switching to Thread 0x7ffff7a006c0 (LWP 39887)]

Thread 2 "prueba1.o" hit Breakpoint 1, greet (argumento=0x0) at prueba1.c:50
50	   printf("Hello, welcome to the program!\n");
(gdb) continue
Continuing.
[New Thread 0x7ffff70006c0 (LWP 39888)]
Hello, welcome to the program!
[Thread 0x7ffff7a006c0 (LWP 39887) exited]
[Switching to Thread 0x7ffff70006c0 (LWP 39888)]

Thread 3 "prueba1.o" hit Breakpoint 2, add (argumento=0x7fffffffdd04) at prueba1.c:60
60	}
(gdb) list
55	void* add(void* argumento) {
56	   float flags = 3.14;
57	   AddArgs* args = (AddArgs*)argumento;
58	   args->result = args->a + args->b;
59	   return NULL;
60	}
(gdb) 
\end{lstlisting}

\subsection{LLDB}{\label{subsec:lldb}}
LLDB es un \gls{depurador} de \gls{código abierto} creado por LLVM en 2003, siendo construido a través de un conjunto de componentes reusables que son ampliamentes usados en librerías existentes de LLVM, como Clang o el \gls{desensamblador} LLVM \cite{LLDB}, lo que hace que tan solo pese en torno a 4 MB.
Aunque LLDB es el depurador por defecto de Xcode en macOS, también está disponible para sistemas Unix y Windows \cite{LLDB}.

LLDB es un \gls{depurador} para programas en C y C++ con una \gls{CLI} y consta de funcionalidades similares a GDB, como colocar puntos de interrupción, inspeccionar la memoria, etc. Por esta razón existe una correspondencia entre las funcionalidades de GDB y LLDB que permite a los usuarios de GDB (\subsectionref{GDB}) cambiar a LLDB sin problemas \cite{LLDB}, aunque pueden existir funcionalidades de GDB que no existan en LLDB. Además, al ser un \gls{depurador} moderno, ofrece funcionalidades como autocompletado de comandos e historial de comandos.

LLDB se caracteriza por tener una \gls{API} de Python más completa que la de GDB, la cual permite desde crear \glspl{script} para automatizar tareas hasta controlar programáticamente la ejecución del programa \cite{LLDBPython}. 

Por lo tanto, LLDB destaca por su integración con LLVM y la posibilidad de controlar el depurador a través de Python, lo que permite a los usuarios crear herramientas personalizadas y adaptar el \gls{depurador} a sus necesidades.

\subsection{RR}{\label{subsec:rr}}
RR es un \gls{depurador} ligero de código abierto creado en 2013 por Mozilla \cite{RRWiki}. Este depurador utiliza GDB para \gls{depurar} programas, sin embargo añade la funcionalidad de rebobinar las ejecuciones de cualquier tipo de programa, siendo uno de ellos un \gls{programa concurrente}, mediante la grabación de la ejecución del programa y la reproducción de esta grabación \cite{RR}.


\subsection{CLion}{\label{subsec:clion}}
CLion es un \gls{IDE} de propiedad de JetBrains, enfocado en la programación en C y C++. Esta aplicación necesita de una licencia para su uso de entre 60€ a 100€, aunque existe una prueba gratuita de 30 días \cite{ClionPrizing}. Cabe destacar que CLion es multiplataforma, disponible para Windows, macOS y Linux y ocupa un espacio de 3.5 GB en disco \cite{ClionDownload}.

CLion incluye una opción para depurar a través de GDB (\subsectionref{gdb}) o LLDB (\subsectionref{lldb}), elegible a través de un menú contextual, mediante una \gls{interfaz gráfica} y extendible con Valgrind y ThreadSanitizer. En este modo depuración se pueden colocar puntos de interrupción, ejecutar el programa, parar el programa, y realizar \textit{\gls{step over}} y \textit{\gls{step into}} \cite{ClionDebugger}.

La depuración muestra multitud de información, organizada en pestañas, como la pila de llamadas, variables locales, variables globales, la salida del programa o inspeccionar la memoria y los \gls{registros} de la \gls{CPU}. También es posible ver el valor de las variables en tiempo real, y modificarlas durante la ejecución del programa \cite{ClionDebuggerToolWindow}.

Por lo que las claves de CLion son su integración de un \gls{depurador} con una \gls{interfaz gráfica}, la variedad de funcionalidades de depuración que permite y la integración con diferentes herramientas de análisis de código.

\subsection{Seer}{\label{subsec:seer}}
Seer es un depurador de código abierto, hecho principalmente en C++ y QTile para la \gls{interfaz gráfica}, que envuelve las funcionalidades de GDB. Este \gls{depurador} solo se encuentra disponible en sistemas Unix y ocupa un espacio de 47.4 MB en disco. 

Seer se caracteriza por tener una \gls{interfaz gráfica} que permite visualizar la información de una forma más amigable y accesible que la \gls{CLI} de GDB. Esta interfaz gráfica tiene una barra de herramientas con botones para las funcionalidades más comunes, como colocar puntos de interrupción, ejecutar el programa, parar el programa, y realizar \textit{\gls{step over}} y \textit{\gls{step into}}. Además, tiene diferentes secciones para visualizar información sobre la pila de llamadas, los distintos \glspl{hilo} en ejecución y las variables locales y globales, pudiendo pulsar en ellos para mostrar más información sobre ellos.

Seer destaca por su ligereza, su \gls{interfaz gráfica}, su rendimiento y la multitud de funcionalidades que ofrece.

\rasterfigure[0.9]{seer.png}{Muestra del depurador de Seer}

\subsection{DDD}{\label{subsec:ddd}}

\section{Otras Herramientas}\label{sec:herramientas}

Además de los \glspl{depurador} de \glspl{programa concurrente}, existen otras herramientas que permiten analizar el comportamiento de un \glspl{programa concurrente} y detectar errores debidos a el uso incorrecto de la \gls{concurrencia}. A continuación se explorarán algunas de estas herramientas.

\subsection{Helgrind}{\label{subsec:helgrind}}

Helgrind es una de las herramientas del grupo Valgrind que detecta errores de sincronización en programas en C y C++ que usan las primitivas POSIX \cite{POSIX}. Estas abstracciones POSIX son: \glspl{hilo} compartiendo un espacio de direccionamiento común, creación de \glspl{hilo}, unión de \glspl{hilo}, salida de \glspl{hilo}, \gls{cerrojo} (mutex), \gls{variable de condición} y \gls{barreras}. Helgrind detecta tres clases de errores: mal uso del API POSIX, potenciales \glspl{interbloqueo}, y \glspl{condicion de carrera} \cite{Helgrind}. 
Estos resultados suelen ser irreproducibles, lo que complica la tarea de encontrar otras maneras para solucionar los errores.

Helgrind monitorea el orden en el que los distintos \glspl{hilo} adquieren el \gls{cerrojo}, permitiendo detectar potenciales \glspl{interbloqueo} y \glspl{condición de carrera}, lo que resulta útil dado que algunos de ellos pueden haberse reproducido durantes las pruebas del programa. Sin embargo, este monitoreo puede ralentizar la ejecución del programa hasta en un factor de 100 \cite{Helgrind}.

Para usar esta herramienta se debe de compilar con la opción \texttt{-pthread} y ejecutar el programa con la opción \texttt{--tool=helgrind}. En el caso en el que Helgrind detecte algún problema en el programa, se mostrará un mensaje de error en la \gls{salida estándar}, esto se puede observar en la siguiente figura.

\begin{lstlisting}[language=bash, caption={Muestra de salida de Helgrind al detectar un error de condición de carrera \cite{HelgrindUc3m}}]
gcc -Wall -g -pthread helgrind_threads_race.c -o helgrind_threads_race valgrind -v  --tool=helgrind  ./helgrind_threads_race
    
==8297== ERROR SUMMARY: 1 errors from 1 contexts (suppressed: 1 from 1)
==8297== 
==8297== 1 errors in context 1 of 1:
==8297== ----------------------------------------------------------------
==8297== 
==8297== Possible data race during write of size 4 at 0xBEE21570 by thread #2
==8297== Locks held: none
==8297==    at 0x8048592: increment_counter (helgrind_threads_race.c:12)
==8297==    by 0x80485C3: counter_thread (helgrind_threads_race.c:20)
==8297==    by 0x402DD35: ??? (in /usr/lib/valgrind/vgpreload_helgrind-x86-linux.so)
==8297==    by 0x405AD4B: start_thread (pthread_create.c:308)
==8297==    by 0x415DB8D: clone (clone.S:130)
==8297== 
==8297== This conflicts with a previous write of size 4 by thread #1
==8297== Locks held: none
==8297==    at 0x8048592: increment_counter (helgrind_threads_race.c:12)
==8297==    by 0x8048682: main (helgrind_threads_race.c:37)
==8297== 
\end{lstlisting}
    
Helgrind se complementa con las funcionalidades tradicionales de los \glspl{depurador} dado que se pueden encontrar errores en \gls{tiempo de compilación} y de forma automática, además de encontrar errores que tal vez aún no se hayan manifestado.

\subsection{ThreadSanitizer}{\label{subsec:thread-sanitizer}}

ThreadSanitizer es una herramienta que detecta \glspl{condición de carrera} para código C y C++. Está formado por una módulo de \gls{inyección de código} y una librería de \gls{tiempo de ejecución}. Esta herramienta puede ralentizar la ejecución en torno a un factor de 5 a 20 e introducir un sobrecarga de memoria de un factor de 5 a 10 \cite{ThreadSanitizer}.

A diferencia de Helgrind (\subsectionref{helgrind}), ThreadSanitizer solo detecta \glspl{condicion de carrera} en \gls{tiempo de ejecución}, lo que significa que si durante la ejecución no se ha recorrido cierto fragmento de código no va poder detectar \glspl{condicion de carrera} que corresponda con ese fragmento \cite{ThreadSanitizerGithub}.

Para usar esta herramienta se debe de compilar con la opción \texttt{-pthread} y ejecutar el programa con la opción \texttt{-fsanitize=thread}. Es importante destacar que esta \gls{compilación} debe de contener todo el código del programa, incluyendo las librerías que se usen, en caso contrario puede detectar \gls{falso positivo}. Tras esta \gls{compilación} la herramienta devolverá en la \gls{salida estándar} los errores que haya detectado, como se muestra en la siguiente figura \cite{ThreadSanitizer}.

\begin{lstlisting}[language=bash, caption={Muestra de salida de ThreadSanitizer al detectar un error de condición de carrera \cite{ThreadSanitizerGithub}}]

g++ simple_race.cc -pthread -fsanitize=thread -o ./a.out 
==================
WARNING: ThreadSanitizer: data race (pid=26327)
  Write of size 4 at 0x7f89554701d0 by thread T1:
    #0 Thread1(void*) simple_race.cc:8 (exe+0x000000006e66)

  Previous write of size 4 at 0x7f89554701d0 by thread T2:
    #0 Thread2(void*) simple_race.cc:13 (exe+0x000000006ed6)

  Thread T1 (tid=26328, running) created at:
    #0 pthread_create tsan_interceptors.cc:683 (exe+0x00000001108b)
    #1 main simple_race.cc:19 (exe+0x000000006f39)

  Thread T2 (tid=26329, running) created at:
    #0 pthread_create tsan_interceptors.cc:683 (exe+0x00000001108b)
    #1 main simple_race.cc:20 (exe+0x000000006f63)
==================

\end{lstlisting}








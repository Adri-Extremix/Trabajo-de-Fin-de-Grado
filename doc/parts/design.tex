\chapter{Diseño}\label{chap:diseno}

En este capítulo se detalla el diseño que tendrá la solución propuesta. Para ello se discutirán sobre las posibles alternativas y cuál de ellas ha sido elegida (\sectionref{estudio-solucion-final}) y posteriormente, se describirá la arquitectura del sistema (\sectionref{arquitectura-sistema}) en base a la solución final elegida.

\section{Construcción de la solución final}\label{sec:estudio-solucion-final}

Para la realización del estudio de la solución final se han tenido en cuenta los requisitos especificados en \sectionref{requisitos} y las herramientas existentes descritas en \chapterref{estado-del-arte}. 

\subsection{Monolítico vs Distribuido} \label{subsec:monolitico-vs-distribuido}

Durante la especificación de los requisitos se han descrito varios requisitos que influyen en la elección del sistema operativo y de la arquitectura del sistema. En concreto los requisitos \sreqref{NF-abstraccion-so} y \sreqref{NF-abstraccion-arch} especifican que la herramienta debe de tener el mismo funcionamiento aunque la arquitectura del sistema o el sistema operativo sean diferentes. Estos requisitos nos obligan a tomar una decisión respecto a la arquitectura del sistema. 

\subsubsection{Monolítico} \label{subsubsec:monolitico}

Una \gls{aplicación monolítica} \cite{DistributedSystems} es aquella que se ejecuta en una única máquina y en un único sistema operativo. Este tipo de aplicaciones son más sencillas de desarrollar y de mantener, ya que no requieren de una comunicación entre diferentes máquinas. Sin embargo, una \gls{aplicación monolítica} no es capaz de ejecutarse en diferentes sistemas operativos, ya que el \gls{código fuente} debe de ser compilado para cada sistema operativo. 

\subsubsection{Distribuido} \label{subsubsec:distribuido}

Una \gls{aplicación distribuida} \cite {DistributedSystems} es aquella que se ejecuta en múltiples \glspl{proceso}. Este tipo de aplicaciones tienen una mayor complejidad de desarrollar, debido a la comunicación entre \glspl{proceso}. Sin embargo, una \gls{aplicación distribuida} que sirva un \gls{servicio web} podría funcionar en cualquier sistema operativo, dado que solo ejecutaría el código del cliente,

\subsection{Elección de la arquitectura} \label{subsec:eleccion-arquitectura}

Tras exponer las ventajas y desventajas de cada una de las arquitecturas, se ha decidido optar por una arquitectura distribuida. La arquitectura monolítica no es una opción viable, ya que para cumplir con los requisitos \sreqref{NF-abstraccion-so}, \sreqref{NF-abstraccion-arch} y \sreqref{NF-multiplataforma} sería necesario compilar el \gls{código fuente} para cada sistema operativo y arquitectura. Esto supondría un gran esfuerzo de desarrollo y mantenimiento, ya que habría que mantener diferentes versiones del \gls{código fuente} para cada sistema operativo y arquitectura.

En cambio, una arquitectura distribuida permitiría tener un único \gls{código fuente} que se ejecute en cualquier sistema operativo y arquitectura, siempre y cuando el cliente sea una \gls{aplicación web}. Esto facilitaría el desarrollo y mantenimiento del \gls{código fuente}, ya que solo habría que mantener una única versión del \gls{código fuente}. Además, la arquitectura distribuida permitiría escalar el sistema de manera más sencilla, ya que se podrían añadir más servidores para soportar más carga de trabajo.
Además, la arquitectura distribuida permitiría ejecutar el \gls{código fuente} en un sistema operativo \gls{Unix} que soporte la ejecución de \glspl{hilo} POSIX, lo que facilitaría la implementación de la funcionalidad de depuración.

\subsection{Almacenamiento de estados vs Depuradores Externos} \label{subsec:depuracion}

\subsubsection{Almacenamiento de estados} \label{subsec:almacenamiento-estados}

Para implementar las operaciones básicas de un \gls{depurador} es necesario moverse por la ejecución del \gls{código fuente}. Es por esta razón que se evaluó la posibilidad de \gls{inyectar código} en el programa cuyo propósito sea almacenar el estado de la ejecución en un momento determinado. Con estos estados almacenados el usuario podría moverse entre cada uno de ellos, facilitando el rebobinado de la ejecución.

Para almacenar estos estados sería necesario introducir código en el programa tras cada \gls{sentencia} que modifique el estado de la ejecución, es decir, creación de variables, modificación de variables o \glspl{llamada al sistema} que generasen nuevos \glspl{hilo}. Lo que supone crear un \textit{\gls{parser}} para encontrar estas \glspl{sentencia} y añadir a continuación una función que almacene la variable y su valor en la función e \gls{hilo} correspondiente.


\subsubsection{Uso de depuradores externos} \label{subsec:depuradores-externos}

También se ha evaluado la opción de utilizar un depurador externo. Durante el \chapterref{estado-del-arte} se analizaron los \glspl{depurador} existentes, dentro de estos \glspl{depurador} tan solo GDB (\subsectionref{gdb}), LLDB (\subsectionref{lldb}) y RR (\subsectionref{rr}) permitían construir una aplicación sobre ellos. Aunque LLDB tenga una \gls{API} en Python que permita la interactividad con la sesión de depuración, carece de la capacidad de rebobinar durante la depuración, tal y como indica el requisito \sreqref {FN-rebobinar}. Por lo tanto, RR es una opción más viable, que aunque no ofrezca una \gls{API} en Python que permita la interactividad, sí que ofrece una interfaz máquina que ofrece una salida formateada.

No obstante, RR registra la ejecución del programa para reproducirla posteriormente, garantizando así una ejecución determinista. Esto limitaría la capacidad del usuario para modificar el flujo de ejecución a su conveniencia. Por ello, se ofrecerá la opción de elegir entre grabar la ejecución, permitiendo el rebobinado, o no grabarla (utilizando solo GDB), permitiendo una ejecución libre de los \glspl{hilo}.

\subsubsection{Elección de depurador} \label{subsec:eleccion-depurador}

Analizadas ambas propuestas, se ha decidido optar por la segunda opción, es decir, utilizar un \gls{depurador} externo. Esta decisión se ha tomado debido a la complejidad que supone implementar un \textit{\gls{parser}} que permita reconocer dónde se debe inyectar código para almacenar el estado de la ejecución. Además, el uso de un \gls{depurador} externo como RR permite aprovechar su funcionalidad de rebobinado, lo que facilita la implementación de las funcionalidades del depurador.

\subsection{Sandbox} \label{subsec:sandbox}

\subsubsection{Entorno no sandbox} \label{subsubsec:entorno-no-sandbox}
Si se optara por no usar un entorno aislado, el código del cliente se ejecutaría en la propia máquina del servidor. Esto podría suponer un problema de seguridad, ya que el código del cliente, que al ser código C puede acceder fácilmente a la memoria, ficheros o recursos, lo que añadido a una intención maliciosa podría provocar un ataque al servidor. Además, el uso de recursos del servidor podría verse comprometido, ya que el código del cliente podría consumir todos los recursos del servidor, afectando a la disponibilidad del mismo.

\subsubsection{Entorno sandbox} \label{subsubsec:entorno-sandbox}
Por otro lado, si se opta por un entorno aislado, el código del cliente se ejecutará en un entorno controlado y limitado. Esto evitaría que el código del cliente pudiera acceder a la memoria o a los ficheros del servidor, lo que aumentaría la seguridad del sistema. 

\subsection{Elección de entorno sandbox} \label{subsec:eleccion-sandbox}
Dadas las ventajas que supone el uso de un entorno aislado y la necesidad de cumplir con  el requisito \sreqref{NF-sandbox}, "\textit{el sistema debe de ejecutar el código en un entorno aislado para evitar problemas de seguridad o disponibilidad}". Para ello, se optó por utilizar \glspl{contenedor} Docker en los que se ejecutarán los programas. De esta manera cualquier código malicioso no podrá acceder a los archivos del sistema ni atacar a la disponibilidad del servidor. Además se limitarán los recursos de los \glspl{contenedor} para controlar el uso de los mismos.

La principal dificultad que supone incorporar \glspl{contenedor} es controlar la disponibilidad de cada uno de ellos. Las características del proyecto nos obligan a que la comunicación entre cliente y docker sea por sesión, lo que implica que durante la sesión el docker no estará disponible para servir a ningún otro cliente. Una posible solución es que la responsabilidad del servidor sea servir los ficheros estáticos del cliente y redirigir los mensajes del cliente al contenedor asociado a la sesión correspondiente y gestionar las comunicaciones entre clientes y \glspl{contenedor}. Esta solución podría generar un \gls{cuello de botella} en el componente servidor en el caso de tener muchos clientes activos, dado que todas las comunicaciones pasarían por el servidor.

\drawiosvgfigure[0.9]{propuesta1_sandbox}{Primera propuesta de arquitectura del sistema}


Es por esta razón que se ha tomado la decisión de quitar responsabilidad al servidor e incorporarla en los \glspl{contenedor}. El servidor se convertirá en un \textit{\gls{proxy}} y se encargará de redirigir la conexión de un cliente a un docker disponible, es decir, solo tendrá información sobre a qué \gls{contenedor} está conectado cada cliente. 
Cada docker alojará un pequeño servidor que sirva los ficheros estáticos y las funcionalidades de los requisitos.

\drawiosvgfigure[0.9]{propuestaFinal_sandbox}{Propuesta final de arquitectura del sistema}

Es importante destacar, que como un \gls{contenedor} puede dejar de estar disponible por un uso malicioso desde el lado cliente, el \gls{proxy} debe de conocer la disponibilidad de cada uno de los \glspl{contenedor}. Para ello cada uno de los \glspl{contenedor} mandará latidos, de manera que el \gls{proxy} pueda conocer qué dockers están disponibles.

Para conseguir todo esto se creará una \gls{imagen Docker} que contenga el código correspondiente a la comunicación tanto con el cliente como con el \gls{proxy}, el código de las funcionalidades del depurador y los ficheros estáticos a servir. Cada uno de estos \glspl{contenedor} se levantará a través de \textit{Docker Compose}, configurado con un \texttt{docker-compose.yml} en el que se indicarán los puertos que cada docker tiene que mapear y las variables de entorno para las comunicaciones. 

Es por esto que la arquitectura distribuida final constará de uno o varios clientes, un único \textit{proxy}, al que se conectarán los clientes para ser redirigidos, y varios \textit{\glspl{contenedor}} que se encargarán de toda la lógica del sistema.

\subsection{Comunicación entre procesos} \label{subsec:comunicacion-procesos}

Dada la existencia de un cliente, un \gls{proxy} y varios \glspl{contenedor}, es necesario establecer una comunicación entre ellos. Esta comunicación se puede realizar de diferentes maneras, pero se han considerado las dos siguientes opciones: 

\begin{itemize}
    \item \textbf{REST}: Permite la comunicación entre \glspl{proceso} a través de peticiones \gls{HTTP}. Esta comunicación es unidireccional, sin control de estados y se establece una conexión por cada petición. Es ideal para controlar una gran cantidad de comunicaciones y para servir unos recursos.
    \item \textbf{Web Sockets}: Esta comunicación es bidireccional, y permite el envío de mensajes en tiempo real. Es ideal para aplicaciones que requieren una comunicación constante.
\end{itemize}

Dadas las características de la aplicación, se optará por usar \gls{REST API} para servir los ficheros estáticos y para enviar los latidos de los \glspl{contenedor}, se usará Web Sockets para la comunicación entre el cliente y su correspondiente \gls{contenedor}, permitiendo entablar la sesión y que exista una comunicación bidireccional.

\subsection{Lenguajes de programación} \label{subsec:programacion}

Dada la arquitectura distribuida del sistema, se necesita usar un lenguaje de programación para el lado cliente y otro para el lado servidor, correspondiente al \gls{proxy} y los \glspl{contenedor}.

\subsubsection{Tecnologías del lado cliente} \label{subsubsec:tecnologias-cliente}

Para la construcción de la aplicación web se ha basará en HTML5 para dar estructura al código, CSS3 para dar estilos, y JavaScript para implementar las distintas funcionalidades. Se ha decidido elegir este \textit{Stack} por ser el más básico y el que más experiencia se tiene. Este \textit{Stack} será más que suficiente dado que el lado cliente no será de gran envergadura y no necesitará más que unos pocos botones, un editor de código y una comunicación a través de WebSockets.


\subsubsection{Tecnologías del lado servidor} \label{sec:tecnologias-servidor}

En cuanto al lenguaje de programación del lado servidor es necesario elegir un lenguaje que permita la implementación de WebSockets para la comunicación en tiempo real, una API REST para servir los ficheros estáticos y enviar los latidos, que trabaje con ficheros para la creación de ficheros con el código del usuario, que permita la concurrencia para poder gestionar \glspl{proceso} en paralelo, y que permita crear un \gls{proceso} en el que ejecutar RR.
Dentro de las opciones destacan Python, Go, Rust y NodeJS, por lo que vamos a analizar cada una de ellas.

\begin{itemize}
    \item \textbf{Python}: Un lenguaje de programación interpretado, muy versátil, aunque con un rendimiento moderado. Gracias a frameworks como Flask para implementar APIs REST, aiohttp o FastAPI para WebSockets, y al módulo threading, para la concurrencia, es una opción viable.
    \item \textbf{Go}: Un lenguaje de programación compilado, con un buen rendimiento y orientado en la concurrencia y el procesamiento en la red. Su soporte nativo para WebSockets mediante goroutines, y frameworks como Gin o Echo para APIs REST lo hacen ideal para este tipo de comunicación.
    \item \textbf{Rust}: Un lenguaje de programación compilado, con un gran rendimiento, aunque con una curva de aprendizaje pronunciada. Tiene bibliotecas como Actix-web para APIs REST y WebSockets, pero no es la más recomendable por su complejidad. 
    \item \textbf{NodeJS}: Un lenguaje de programación interpretado con excelente soporte tanto para APIs REST (Express.js) como para WebSockets (Socket.io), aunque su concurrencia se basa en un solo hilo utilizando un bucle de eventos y un modelo asíncrono; lo que no es un paralelismo real. Su modelo no es ideal para operaciones intensivas en CPU.
\end{itemize}

Tanto Go como Python son lenguajes adecuados para el proyecto. Aunque Go ofrece un rendimiento superior, la disponibilidad de la biblioteca Python \textit{pygdbmi}, que facilita la gestión de una sesión RR/GDB y la obtención de la salida MI en formato JSON, convierte a Python en la opción más conveniente.

\section{Actualizacion de los requisitos del sistema} \label{sec:actualizacion-requisitos}
Tras las decisiones tomadas durante la construcción de la solución final, se han actualizado los requisitos del sistema. Las decisiones tomadas han llevado a generar 3 nuevos requisitos funcionales de software y un nuevo requisito no funcional, que se describen a continuación:

\begin{softwareReq}{FN}{arquitectura-distribuida}
    {pc=h,pd=h,s=c,v=h}
    {CA-sistema}
    El sistema debe implementarse como una aplicación distribuida, donde los componentes puedan ejecutarse en diferentes ubicaciones físicas, permitiendo que el código del usuario se ejecute en un entorno controlado separado del cliente.
\end{softwareReq}

\begin{softwareReq}{FN}{aplicacion-web}
    {pc=h,pd=h,s=c,v=h}
    {CA-sistema}
    El sistema debe implementarse como una aplicación web accesible desde un navegador, permitiendo su uso sin instalación específica en la máquina del usuario y garantizando la compatibilidad multiplataforma.
\end{softwareReq}

\begin{softwareReq}{FN}{comunicacion-websockets}
    {pc=m,pd=h,s=c,v=m}
    {CA-visibilidad-hilos, CA-visibilidad-variables, CA-evolucion-hilos}
    El sistema debe utilizar WebSockets para la comunicación en tiempo real entre el cliente y el servidor, permitiendo la actualización inmediata de la información de depuración sin necesidad de recargar la página.
\end{softwareReq}

\begin{softwareReq}{NF}{docker}
    {pc=h,pd=h,s=c,v=h}
    {RE-sandbox}
    El sistema debe ejecutarse en un entorno aislado utilizando contenedores Docker, garantizando la seguridad y el control de recursos durante la ejecución del código del usuario.
\end{softwareReq}

Tras la actualización de los requisitos, se han actualizado los requisitos de trazabilidad, esto se muestra en las siguientes tablas \tabref{trazabilidad2FN-CA} \tabref{trazabilidad2NF-RE}.


\begin{table}[htb]
    \ttabbox[\FBwidth]
      {\caption{Trazabilidad de los requisitos funcionales con los requisitos de capacidad}\label{tab:trazabilidad2FN-CA}}
      {\traceabilityFNCA}
  \end{table}

\begin{table}[htb]
    \ttabbox[\FBwidth]
      {\caption{Trazabilidad de los requisitos no funcionales con los requisitos de restricción}\label{tab:trazabilidad2NF-RE}}
      {\traceabilityNFRE}
  \end{table}

\FloatBarrier

\section{Arquitectura del sistema}\label{sec:arquitectura-sistema}

\subsection{Componentes del sistema} \label{sec:componentes-sistema}

Tal y como se ha comentado en \sectionref{estudio-solucion-final}, el sistema consta de tres partes: el cliente, el \textit{\gls{proxy}} y el contenedor docker.

En la figura \ref{fig:componentes_TFG} se muestra el diagrama de componentes UML \cite{Cook2017} donde se pueden observar los distintos componentes y sus diferentes interfaces.

\drawiosvgfigure[0.9]{componentes_TFG}{Diagrama de componentes del sistema}

A continuación se describen los distintos componentes del sistema y sus responsabilidades.

\begin{itemize}
    \item \textbf{Cliente}: Permite facilitar la interacción del usuario con el depurador, además de visualizar las salidas y cambios ocurridos durante la ejecución. En este componente se podrá escribir código, y comunicará las acciones al servidor para ejecutar las correspondientes funcionalidades. 
    Dentro de la aplicación web, existen dos componentes principales:

    \begin{itemize}
        \item \textbf{Interfaz Gráfica}: Su responsabilidad es actualizar la información que se muestra por pantalla.
        \item \textbf{Comunicación con el contenedor (Web Socket)}: Se encarga de enviar y recibir mensajes del contenedor.
        \item \textbf{Editor de Código}: Su responsabilidad es permitir al usuario escribir el código que se va a depurar.
    \end{itemize}

    \item \textbf{Proxy}: Se encargará de gestionar la disponibilidad de los \glspl{contenedor} y los \glspl{contenedor} libres. Además redirigirá los clientes a los \glspl{contenedor} correspondientes.
    Este componente consta de dos componentes:
    \begin{itemize}
        \item \textbf{Gestión de \glspl{contenedor}}: Al recibir una petición de un cliente se le redirigirá a un contenedor libre, este contenedor será asociado a este cliente hasta que se desconecte, por lo que dejará de estar libre.
        \item \textbf{Disponibilidad de \glspl{contenedor}}: Alojará un servicio para escuchar los \glspl{contenedor} disponibles. Si tras cierto tiempo deja de recibir latidos de un contenedor lo dará por muerto, por lo que no podrá ser asociado a nuevos clientes.  
    \end{itemize}

    \item \textbf{Contenedor Docker}: Se encarga de servir los ficheros estáticos al cliente y de ejecutar las funcionalidades pedidas por este.
    \begin{itemize}
        \item \textbf{Servicio de ficheros estáticos}: El contenedor servirá el html, css y JavaScript del cliente al recibir una petición GET por parte del cliente.
        \item \textbf{Web Socket Docker}: El contenedor utilizará Web Socket para la comunicación con el cliente.
        \item \textbf{Funcionalidades de Compilación}: Este componente se encargará de compilar el código recibido y ejecutarlo sin modo depuración.
        \item \textbf{Funcionalidades de Depuración}: Este componente contendrá la implementación de las funcionalidades de depuración, las cuales serán llamadas a través de la API.
        \item \textbf{Envío de latidos}: El contenedor enviará latidos cada cierto tiempo.
    \end{itemize}
\end{itemize}


Para la especificación de los distintos componentes se ha usado la siguiente plantilla: 

\printcomptemplate
% ------- Lado Cliente -------
\begin{component}{Interfaz Gráfica}
{Mostrar la información actualizada al usuario por pantalla}
{COM-Editor de Código, COM-Web Socket Cliente} % dependencias
{Estado actual del sistema} % in-data
{NA} % out-data
{FN-GUI, FN-aviso-error-concurrencia, FN-visualizacion-variables, FN-visualizacion-hilos, FN-visualizacion-evolucion, FN-aplicacion-web} % relations
La Interfaz Gráfica debe de gestionar la interacción del usuario con la aplicación y mostrar la información de manera estructurada y clara. %description 
\end{component}

\begin{component}{Editor de Código}
    {Permitir al usuario escribir el código que se va a depurar}
    {\NA} % dependencias
    {\NA} % in-data
    {Código escrito, estado del editor de código} % out-data
    {FN-GUI, FN-editor, FN-editor-breakpoint} % relations
    Este componente permitirá al usuario escribir código, colorear la sintaxis y gestionar de forma visual los breakpoints. %description
\end{component}

\begin{component}{Web Socket Cliente}
{Punto de comunicación por Web Socket desde el cliente al servidor}
{COM-Editor de Código, COM-Web Socket y REST API Servidor} % dependencias
{Mensajes del servidor} % in-data
{Cambios del sistema ,Mensajes del cliente} % out-data
{FN-comunicacion-websockets, FN-arquitectura-distribuida} % relations
La comunicación con el servidor se encargará de gestionar la comunicación entre el cliente y el servidor a través de Web Sockets. %description
\end{component}

% ------- Lado Proxy -------

\begin{component}{Gestión de contenedores}
{Asignar clientes a contenedor libres}
{COM-Disponibilidad de contenedores} % dependencias
{Contenedores Disponibles} % in-data
{URL para la conexión} % out-data
{FN-comunicacion-websockets, FN-arquitectura-distribuida} % relations
Este componente asignará a un nuevo cliente un contenedor libre hasta que se desconecte. Al asignar a un contenedor le redireccionará al contenedor correspondiente %description
\end{component}

\begin{component}{Disponibilidad de contenedores}
{Escucha latidos de los contenedores para comprobar su disponibilidad }
{COM-Gestión de contenedores, COM-Envío de Latidos} % dependencias
{Latido} % in-data
{Contenedores Disponibles} % out-data
{FN-arquitectura-distribuida, NF-docker} % relations
Registra cada uno de los latidos escuchados por cada contenedor, si tras 10 segundos no ha recibido un latido dará por no disponible al contenedor. %description
\end{component}

% ------ Lado Contenedor -------

\begin{component}{Web Socket Docker}
{Gestión de comunicación entre los contenedores y el cliente a través de Web Sockets}
{COM-Web Socket Cliente, COM-Funcionalidades Depuración} % dependencias
{Mensajes del cliente} % in-data
{Mensajes del contenedor} % out-data
{\NA} % relations
La comunicación con los contenedores se encargará de gestionar la comunicación entre el contenedor y el cliente a través de Web Sockets. %description 
\end{component}

\begin{component}{Funcionalidades Compilación}
{Se encarga de compilar el código recibido y ejecutarlo sin modo depuración}
{COM-Web Socket Docker}
{Código}
{Salida Estándar}
{FN-fichero-codigo, FN-compilacion, FN-ejecutar, FN-aviso-error-concurrencia, NF-docker}
Este componente se encarga de realizar acciones básicas que no están relacionadas directamente con la depuración, como compilar el código y ejecutarlo.
\end{component}

\begin{component}{Funcionalidades de Depuración}
{Gestionar la ejecución de las funcionalidades de depuración}
{COM-Web Socket Docker} % dependencias
{Funcionalidad a realizar} % in-data
{Estado tras la ejecución de la funcionalidad} % out-data
{FN-depurar, FN-step-over, FN-step-into, FN-step-out, FN-continuar-ejecucion, FN-rebobinar} % relations
Este componente ejecutará las funcionalidades del depurador y devolverá el resultado de aplicar esas funcionalidades. %description
\end{component}

\begin{component}{Envío de Latidos}
{Envío de latidos}
{COM-Disponibilidad de contenedores} % dependencias
{\NA} % in-data
{Latido} %out-data
{FN-arquitectura-distribuida, NF-docker} % relations
Envío de latidos cada 5 segundos a la dirección del \textit{\gls{proxy}}
\end{component}

\begin{component}{Servicio de Ficheros Estáticos}
{Envío de ficheros estáticos al cliente}
{\NA}
{Ficheros HTML, CSS y JavaScript}
{FN-aplicacion-web, NF-docker} %relations
{Se encarga de enviar ficheros estáticos tras recibir una petición GET por parte del cliente para que este pueda acceder a su aplicación web.}
\end{component}

\FloatBarrier


\begin{landscape}
    \subsubsection{Trazabilidad} {\label{subsubsec:trazabilidad-comp}}
    \begin{table}[htb]
        \ttabbox[\FBwidth]
        {\caption{Trazabilidad de los componentes con los requisitos funcionales}\label{tab:trazabilidadCOM-FN}}
        {\traceabilityCompFN}
    \end{table}
\end{landscape}


\subsection{Interfaz gráfica del sistema} \label{sec:interfaz-grafica}
La interfaz gráfica del sistema se ha diseñado para facilitar la interacción del usuario con el depurador. La figura \figref{Mockup} muestra un mockup de la interfaz gráfica.

% \svgfigure[width]{filename}{caption}
  \begin{figure}[H]
    \ffigbox[\FBwidth]
      {%
        \caption{Mockup de la interfaz gráfica del sistema} 
        \label{fig:Mockup}
      }%
      {\includesvg[inkscapelatex=false,width=0.95\textwidth]{Mockup.svg}}
  \end{figure}

La interfaz gráfica se divide en dos secciones:
\begin{itemize}
    \item \textbf{Editor de Código}: Permite al usuario escribir el código que se va a \gls{depurar}. Este editor cuenta con resaltado de sintaxis y permite establecer puntos de interrupción (\glspl{breakpoint}) en el código. Justo a la derecha se encuentran dos botones, para compilar (obligatorio para acceder al resto de funcionalidades) y ejecutar. En la parte inferior del editor se muestra la salida estándar del código, lo que permite al usuario ver los resultados de la \gls{compilación} y ejecución del código.
    \item \textbf{Panel de Depuración}: En esta sección se mostrará la información de los \glspl{hilo} y las variables del programa. En la parte superior se tendrán botones para realizar las diferentes acciones de depuración. En la parte izquierda se mostrán los \glspl{hilo} activos, y en la sección de código en la que se han encontrado al pararse. En la parte derecha se mostrán las variables de cada \gls{hilo}, junto con su valor actual. Si se clicka en una variable global se mostrará la evolución de su valor a lo largo de la ejecución del programa.  
\end{itemize}

\subsection{Diagrama de secuencia del sistema} \label{sec:diagrama-secuencia}
Con el objetivo de visualizar la interacción entre los distintos componentes del sistema, se han desarrollado varios diagramas de secuencia que muestran la interacción entre los distintos componentes del sistema.

\subsubsection{Diagrama de secuencia de la conexión del cliente} \label{subsec:diagrama-secuencia-conexion}

En la figura \ref{fig:Conexion_contenedor} se muestra cómo el cliente realiza una petición \texttt{GET} al \textit{\gls{proxy}}, el cual le redirige a un \gls{contenedor} libre, devolviendo este los ficheros estáticos al cliente.

\svgfigure[0.6]{Conexion_contenedor}{Diagrama de secuencia de la conexión del cliente al contenedor}

\FloatBarrier

\subsubsection{Diagrama de secuencia de la gestion de contenedores} \label{subsec:diagrama-secuencia-gestion-contenedores}

En la figura \ref{fig:Gestion_contenedores} se muestra cómo los \glspl{contenedor} envían latidos al \textit{\gls{proxy}} con cierta frecuencia, y cómo el \textit{\gls{proxy}} gestiona la disponibilidad de los \glspl{contenedor}. Si un \gls{contenedor} deja de enviar latidos durante un tiempo determinado, el \textit{\gls{proxy}} lo marca como no disponible.
\svgfigure[0.55]{Gestion_contenedores}{Diagrama de secuencia de la gestión de contenedores}

\FloatBarrier

\subsubsection{Diagrama de secuencia de la compilación y ejecución} \label{subsec:diagrama-secuencia-compilacion-ejecucion}
En la figura \ref{fig:Compilacion_ejecucion} se muestra cómo el cliente envía una petición de \gls{compilación} al \gls{contenedor}, enviando el código a compilar. El \gls{contenedor} compila el código y devuelve la salida estándar de esa \gls{compilación} al cliente.
Tras esto el cliente envía una petición de ejecución al \gls{contenedor}, el cual ejecuta el código y devuelve la salida estándar de esa ejecución al cliente.
\svgfigure[0.55]{Compilacion_ejecucion}{Diagrama de secuencia de la \gls{compilación} y ejecución del código}

\FloatBarrier

\subsubsection{Diagrama de secuencia de la depuración} \label{subsec:diagrama-secuencia-depuracion}
En la figura \ref{fig:Depuracion} se muestra un posible flujo de depuración. El cliente compila el código, mandando el código y los breakpoints al \gls{contenedor}. Este contesta con la salida de la \gls{compilación}.
Tras esto el cliente envía una petición de inicio de depuración al \gls{contenedor}, y el \gls{contenedor} inicia la depuración hasta el primer breakpoint. El cliente recibe la información de los \glspl{hilo} y las variables, y muestra la información al usuario.
Tras esto el cliente manda un \texttt{step} y el \gls{contenedor} ejecuta la siguiente instrucción, actualizando la información de los \glspl{hilo} y las variables.
Para finalizar la depuración, el cliente envía una petición de finalización de depuración al \gls{contenedor}, el cual finaliza la sesión de depuración y devuelve la información final al cliente.
\svgfigure[0.6]{Depuracion}{Diagrama de secuencia de la depuración del código}

\FloatBarrier

\subsubsection{Estructura de los mensajes de depuración} \label{subsec:estructura-mensajes-depuracion}
Tal y como se ha desarrollado en \subsectionref{diagrama-secuencia-depuracion}, durante la depuración los \glspl{contenedor} responden a las funcionalidades enviando un mensaje que contienen la información de los \glspl{hilo} y las variables. Esta información se almacena en los \glspl{contenedor} en diccionarios de Python, y se envía al cliente en formato \gls{JSON}. Y la estructura es la siguiente:

\begin{figure}[htb]
    \ffigbox[\FBwidth]
    {%
    \caption{Estructura de los mensajes de depuración}
    \label{fig:estructura-mensajes-depuracion}
    }
    {
    \begin{tcolorbox}
        \dirtree{%
        .1 Hilos. 
            .2 Hilo-1:.
                .3 function: "main".
                .3 line: "10".
                .3 code: "int main(){ int a = 5;...}".
                .3 variables:. 
                    .4 a: 5.
                    .4 b: 10.
            .2 Hilo-2:.
                .3 function: "funcion1".
                .3 line: "47".
                .3 code: "void funcion1(){ int c = 1;...}".
                .3 variables:.
                    .4 b: 10. 
                    .4 c: 1.
            .2 Hilo-3:.
                .3 function: "funcion2".
                .3 line: "73".
                .3 code: "void funcion2(){ int d = 2;...}".
                .3 variables:.
                    .4 d: 2.
        }
    \end{tcolorbox}
    }
\end{figure}

Es importante destacar que los \glspl{hilo} se actulizan según el estado de la ejecución, de manera que si un \gls{hilo} no se ha ejecutado o ha finalizado, no se incluirá en el mensaje.

De esta manera, el cliente puede visualizar la información de los \glspl{hilo} y las variables de manera clara y estructurada. Además, se pueden añadir nuevas funcionalidades que envíen información adicional al cliente, como la información de los \glspl{breakpoint} o el estado de la ejecución.
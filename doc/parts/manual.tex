\chapter{Manual de Usuario}\label{app:manual}

Este apéndice contiene el manual de usuario para la herramienta de depuración de programas concurrentes desarrollada como parte de este Trabajo de Fin de Grado.

La \textit{Herramienta Didáctica para la Programación Concurrente} es una aplicación web que facilita el aprendizaje y comprensión de los conceptos de concurrencia, permitiendo a estudiantes y profesores visualizar el comportamiento de programas multihilo en tiempo real.

La herramienta permite ejecutar y depurar código C con múltiples hilos, observando visualmente el comportamiento de cada hilo y el estado de las variables durante la ejecución, lo que facilita la comprensión de conceptos complejos relacionados con la programación concurrente.

\section{Requisitos del sistema}

Para ejecutar esta herramienta, es necesario contar con:
\begin{itemize}
    \item \textbf{Docker} y \textbf{Docker Compose}: Para gestionar los contenedores aislados
    \item \textbf{Node.js} (versión 14 o superior): Para la construcción del frontend
\end{itemize}

\subsection{Instalación de requisitos previos}

Antes de poder utilizar la herramienta, es necesario instalar los siguientes componentes en su sistema operativo:

\subsubsection{Instalación de Docker}

Docker es esencial para el funcionamiento de la herramienta, ya que proporciona los entornos aislados donde se ejecutará el código.

\begin{itemize}
    \item \textbf{Linux (Ubuntu/Debian)}:
    \begin{verbatim}
    sudo apt update
    sudo apt install docker.io docker-compose
    sudo systemctl enable --now docker
    sudo usermod -aG docker $USER
    \end{verbatim}
    
    \item \textbf{macOS}:
    Descargar e instalar Docker Desktop desde: \url{https://www.docker.com/products/docker-desktop/}
    
\end{itemize}

\subsubsection{Instalación de Node.js}

Node.js es necesario para construir y ejecutar el frontend de la aplicación.

\begin{itemize}
    \item \textbf{Linux (Ubuntu/Debian)}:
    \begin{verbatim}
    sudo apt install -y nodejs
    \end{verbatim} 
    
    \item \textbf{macOS} (usando Homebrew):
    \begin{verbatim}
    brew install node
    \end{verbatim}
    
\end{itemize}

\section{Instalación y ejecución}

\subsection{Obtención del código fuente}

Para comenzar, se debe clonar el repositorio de GitHub:

    \begin{verbatim}
    git clone https://github.com/Adri-Extremix/Trabajo-de-Fin-de-Grado
    \end{verbatim}

\subsection{Preparación de scripts}

Es necesario dar permisos de ejecución a los scripts del sistema:

    \begin{verbatim}
    cd Trabajo-de-Fin-de-Grado
    chmod +x start_proyect.sh
    chmod +x close_proyect.sh
    \end{verbatim}

\subsection{Inicio de la aplicación}

Para iniciar todos los componentes del sistema:

    \begin{verbatim}
    ./start_proyect.sh
    \end{verbatim}

Este script realizará las siguientes acciones:
\begin{itemize}
    \item Detectar la dirección IP local
    \item Crear una red Docker para la comunicación entre contenedores
    \item Construir el frontend
    \item Iniciar el proxy
    \item Lanzar los contenedores de depuración
    \item Abrir el navegador con la URL del sistema
\end{itemize}

La aplicación estará accesible automáticamente en el navegador, o alternativamente en la dirección \texttt{http://<IP\_LOCAL>:8080}.

\subsection{Cierre de la aplicación}

Para detener todos los servicios cuando se termine de utilizar la herramienta:

\begin{verbatim}
    ./close_proyect.sh
\end{verbatim}

\section{Guía de uso}

\subsection{Características principales}

La herramienta incluye las siguientes funcionalidades:

\begin{itemize}
    \item \textbf{Ejecución de código C}: Compila y ejecuta programas C con múltiples hilos
    \item \textbf{Depuración interactiva}: Utiliza comandos de depuración como step-in, step-over, y breakpoints
    \item \textbf{Visualización en tiempo real}: Observa el estado y la evolución de los hilos durante la ejecución
    \item \textbf{Entorno seguro}: El código se ejecuta en contenedores Docker aislados
    \item \textbf{Arquitectura escalable}: Múltiples contenedores Docker pueden servir a diferentes usuarios
    \item \textbf{Comunicación WebSockets}: Actualización en tiempo real del estado de la depuración
\end{itemize}

\subsection{Interfaz principal}

La interfaz de la aplicación está dividida en dos secciones:
\begin{itemize}
    \item \textbf{Sección de código}: Donde se escribe y edita el código C
    
    \begin{itemize}
        
        \item \textbf{Editor de código}: Donde se escribe el programa C
        \item \textbf{Panel de control}: Con botones para compilar y ejecutar
        \item \textbf{Terminal de salida}: Muestra la salida del programa y mensajes del sistema

    \end{itemize}

    \item \textbf{Sección de depuración}: Donde se gestionan los puntos de interrupción y se observa el estado de la ejecución

    \begin{itemize}
            
        \item \textbf{Panel de depuración}: Contiene botones para iniciar, detener y controlar la depuración
        \item \textbf{Vista de variables}: Muestra las variables activas y sus valores
        \item \textbf{Vista de hilos}: Visualiza el estado de los diferentes hilos

    \end{itemize}
    
\end{itemize}

\subsection{Flujo de trabajo básico}

\begin{enumerate}
    \item Escribir o cargar código C en el editor de la interfaz web
    \item Añadir puntos de interrupción (breakpoints) haciendo clic junto al número de línea
    \item Compilar el código con el botón \textit{Compilar}
    \item Si hay errores de compilación, corregirlos y volver a compilar
    \item Para ejecución normal, presionar \textit{Ejecutar}
    \item Para depuración, cambiar a la sección de depuración
    \item Iniciar la depuración con el botón \textit{Run}
    \item Utilizar los controles de depuración:
    \begin{itemize}
        \item \textbf{Step Over}: Avanza una instrucción sin entrar en funciones
        \item \textbf{Reverse Step Over}: Retrocede una instrucción sin entrar en funciones
        \item \textbf{Step Into}: Avanza entrando en las funciones llamadas
        \item \textbf{Reverse Step Into}: Retrocede entrando en las funciones llamadas
        \item \textbf{Step Out}: Sale de la función actual
        \item \textbf{Reverse Step Out}: Retrocede saliendo de la función actual
        \item \textbf{Continue}: Ejecuta hasta el siguiente breakpoint
        \item \textbf{Reverse Continue}: Retrocede hasta el siguiente breakpoint
        \item \textbf{Stop}: Detiene la ejecución del programa

    \end{itemize}
    \item Observar el estado de variables y la evolución de los hilos en tiempo real
    \item La depuración puede detenerse en cualquier momento con \textit{Stop}
\end{enumerate}

\begin{tcolorbox}[colback=yellow!10!white,colframe=red!50!black,title=\textbf{Importante}]
Cada vez que se realicen cambios en el código o en los puntos de interrupción, es necesario recompilar el código para que los cambios surtan efecto.
\end{tcolorbox}


\chapter{Declaración de Uso de IA Generativa en el Trabajo de Fin de Grado}
\label{chap:IA}
\section*{He usado IA Generativa en este trabajo}
Marca lo que corresponda:
\begin{tabular}{|p{3cm}|c|c|}
\hline
& SÍ & NO \\
\hline
Marque con X: & \multicolumn{1}{|c|}{X} & \multicolumn{1}{|c|}{} \\
\hline
\end{tabular}

\section*{Parte 1: Reflexión sobre Comportamiento Ético y Responsable}

\begin{table}[H]
\centering
\begin{tabular}{|p{4cm}|p{4cm}|p{4cm}|}
\hline
\multicolumn{3}{|c|}{\textbf{Pregunta}} \\ \hline
\multicolumn{3}{|p{12cm}|}{En mi interacción con herramientas de IA Generativa he remitido datos de carácter sensible con la debida autorización de los interesados} \\ \hline
SÍ, he usado estos datos con autorización & NO, he usado estos datos sin autorización & \textbf{NO, no he usado datos de carácter sensible} \\ \hline
\multicolumn{3}{|p{12cm}|}{En mi interacción con herramientas de IA Generativa he remitido materiales protegidos por derechos de autor con la debida autorización de los interesados} \\ \hline
SÍ, he usado estos materiales con autorización & NO, he usado estos materiales sin autorización & \textbf{NO, no he usado materiales protegidos por derechos de autor} \\ \hline
\multicolumn{3}{|p{12cm}|}{En mi interacción con herramientas de IA Generativa he remitido datos de carácter personal con la debida autorización de los interesados} \\ \hline
SÍ, he usado estos datos con autorización & NO, he usado estos datos sin autorización & \textbf{NO, no he usado datos de carácter personal} \\ \hline
\multicolumn{3}{|p{12cm}|}{Mi utilización de la herramienta de IA Generativa ha respetado sus términos de uso, así como los principios éticos esenciales, no orientándola de manera maliciosa a obtener un resultado inapropiado para el trabajo presentado, es decir, que produzca una impresión o conocimiento contrario a la realidad de los resultados obtenidos, que suplante mi propio trabajo o que pueda resultar en un perjuicio para las personas} \\ \hline
\textbf{SÍ} & NO & \\ \hline
\end{tabular}
\caption{Declaración de comportamiento ético en el uso de IA Generativa}
\end{table}

\section*{Parte 2: Declaración de Uso Técnico}

\noindent Declaro haber hecho uso del sistema de IA Generativa \textit{ChatGPT} y \textit{Copilot} para: 

\subsubsection*{Documentación y redacción:}
\begin{itemize}
    \item \textbf{Soporte a la reflexión en relación con el desarrollo del trabajo: proceso iterativo de análisis de alternativas y enfoques utilizando la IA} 
    \begin{itemize}
        \item Estudio de alternativas, viabilidad y enfoques para el desarrollo del trabajo.
    \end{itemize}
    \item \textbf{Revisión o reescritura de párrafos redactados previamente} 
    \begin{itemize}
        \item He pedido la revisión de párrafos ya redactados para mejorar su claridad y calidad. 
        \item Redacción de ciertas descripciones para el glosario de términos.
    \end{itemize}
    \item \textbf{Búsqueda de información o respuesta a preguntas concretas} 
    \begin{itemize}
        \item Se ha solicitado información específica sobre arquitecturas distribuidas, Docker y comunicación entre contenedores.
    \end{itemize}
\end{itemize}

\subsubsection*{Desarrollar contenido específico}
Se ha hecho uso de IA Generativa como herramienta de soporte para el desarrollo del contenido específico del TFG, incluyendo:
\begin{itemize}
    \item \textbf{Asistencia en el desarrollo de líneas de código (programación)} 
    \begin{itemize}
        \item Se ha desarrollado el código a través del asistente de IA Generativa \textit{Copilot}.
        \item Se ha compartido el código para encontrar errores y solicitar soluciones a problemas específicos.
        \item Se ha usado el modo \textit{Agente} de \textit{Copilot} para el desarrollo de la parte del cliente, sobre todo para el desarrollo de la interfaz de usuario.
        \begin{itemize}
            \item Empleando el prompt: \textit{"Quiero tener una interfaz con dos secciones, la parte izquierda contrendrá los fragmentos de código de cada uno de los hilos y en la parte derecha las variables de cada uno de los hilos. Esta información será recibida por parte del backend."} Teniendo como intención: \textit{"Desarrollar una interfaz de usuario básica de la que partir"}
    \end{itemize}
    \item \textbf{Procesos de optimización} 
    \begin{itemize}
        \item He solicitado optimizaciones en el código para mejorar su rendimiento y eficiencia.
        \item He pedido sugerencias para mejorar la estructura del código y su legibilidad.
        \item He utilizado autocompletado de código para acelerar el proceso de desarrollo. 
    \end{itemize}
    \item \textbf{Inspiración de ideas en el proceso creativo} 
    \begin{itemize}
        \item He contextualizado el trabajo y he solicitado sugerencias para mejorar la arquitectura distribuidas y la comunicación entre contenedores.
        \begin{itemize}
            \item Empleando el prompt: \textit{"Teniendo un sistema distribuido a través de contenedores Docker y una aplicación web, ¿Qué comunicación debería de usar entre los contenedores y entre la aplicación web y los contenedores?"} Teniendo como intención: \textit{"Analizar las principales alternativas de comunicación y cuáles son sus ventajas y desventajas en cada caso."}
        \end{itemize}
    \end{itemize}
    \end{itemize}

\end{itemize} 

\section*{Parte 3: Reflexión sobre Utilidad}
La IA Generativa es una herramienta más, que permite optimizar el proceso de aprendizaje y desarrollo de un proyecto. La IA Generativa acelera el proceso de búsqueda de información, permite fomentar la creatividad y mejora de la calidad de las ideas y del código, pero no sustituye el proceso de aprendizaje y reflexión que debe realizar el estudiante. Como cualquier otra herramienta, su uso debe de ser responsable, contrastar la información obtenida y no generar una dependencia excesiva de la misma.

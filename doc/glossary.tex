\makeglossaries
% \makenoidxglossaries  % slower, but no need to do $ makeglossaries report.tex


% usage:
% - \gls{term}: regular stuff
% - \Gls{term}: first uppercase
% - \glspl{term}: plural
% - \Glspl{term}: you know the drill
% - \glsdisp{term}{custom text}: link with custom text
% https://www.overleaf.com/learn/latex/Glossaries

\newglossaryentry{depurar}{
    name = {depurar},
    description = {Proceso de identificar y corregir errores en un programa},
    see = {depurador}
}


\newglossaryentry{depurador} {
    name = {depurador},
    description = {Programa que tiene como objetivo probar y depurar otros programas, a través del control de la ejecución del programa y la inspección del estado del mismo},
    plural = {depuradores},
    see = {proceso}
}

\newglossaryentry{programa concurrente} {
    name = {programa concurrente},
    description = {Programa que utiliza múltiples procesos en ejecución ,que pueden ser ejecutados en paralelo o de forma intercalada, para realizar tareas de forma simultánea},
    plural = {programas concurrentes},
    see = {proceso}
}

\newglossaryentry{ensamblador}{
    name = {ensamblador},
    description = {Lenguaje de programación de bajo nivel que representa instrucciones en código máquina}
}

\newglossaryentry{planificador}{
    name = {planificador},
    description = {Software responsable de gestionar y asignar recursos a los procesos en un sistema operativo para optimizar su ejecución},
    see = {proceso}
}

\newglossaryentry{proceso} {
    name = {proceso},
    description = {Entidad que representa una tarea en ejecución en un sistema operativo},
    plural = {procesos}
}

\newglossaryentry{step over} {
    name = {step over},
    description = {Comando de depuración que permite avanzar una instrucción en el código fuente sin entrar en las funciones que se llaman},
    see = {depurar}
}

\newglossaryentry{step into} {
    name = {step into},
    description = {Comando de depuración que permite avanzar una instrucción en el código fuente y entrar en las funciones que se llaman},
    see = {depurar}
}

\newglossaryentry{interfaz gráfica} {
    name = {interfaz gráfica},
    description = {Tipo de interfaz que permite la interacción con un programa a través de elementos visuales, como botones, menús y ventanas}
}

\newglossaryentry{Ley de Moore}{
    name = {Ley de Moore},
        description = {Principio que establece que el número de transistores en los circuitos integrados se duplica aproximadamente cada dos años, lo que conduce a un crecimiento exponencial en el rendimiento de los dispositivos electrónicos}
}

\newglossaryentry{script} {
    name = {script},
    description = {Programas que automatizan tareas repetitivas},
    plural = {scripts}
}

\newglossaryentry{Unix} {
    name = {Unix},
    description = {Familia de sistemas operativos multiusuario y multitarea, desarrollados en los años 70}
}

\newglossaryentry{C}{
    name = {C},
    description = {Lenguaje de programación de propósito general, creado por Dennis Ritchie en los años 70}
}

\newglossaryentry{C++}{
    name = {C++},
    description = {Lenguaje de programación de propósito general, derivado de C, creado por Bjarne Stroustrup en los años 80}
}

\newglossaryentry{D}{
    name = {D},
    description = {Lenguaje de programación de propósito general, creado por Walter Bright en los años 2000}
}

\newglossaryentry{Fortran}{
    name = {Fortran},
    description = {Lenguaje de programación de propósito general, creado por IBM en los años 50}
}

\newglossaryentry{Go}{
    name = {Go},
    description = {Lenguaje de programación de propósito general, creado por Google en los años 2000}
}

\newglossaryentry{Pascal}{
    name = {Pascal},
    description = {Lenguaje de programación de propósito general, creado por Niklaus Wirth en los años 70}
}

\newglossaryentry{Rust}{
    name = {Rust},
    description = {Lenguaje de programación de propósito general, creado por Mozilla en los años 2010}
}

\newglossaryentry{Python}{
    name = {Python},
    description = {Lenguaje de programación de alto nivel, creado por Guido van Rossum en los años 90}
}

\newglossaryentry{CLion} {
    name = {CLion},
    description = {IDE de propiedad de JetBrains, enfocado en la programación en C/C++}
}

\newglossaryentry{Valgrind} {
    name = {Valgrind},
    description = {Herramienta de análisis de código que permite detectar errores de memoria y rendimiento en programas}
}

\newglossaryentry{ThreadSanitizer} {
    name = {ThreadSanitizer},
    description = {Herramienta de análisis de código que permite detectar errores de concurrencia en programas}
}

\newglossaryentry{LLVM} {
    name = {LLVM},
    description = {Proyecto de compilador modular y herramientas relacionadas, desarrollado por la Universidad de Illinois en Urbana-Champaign}
}

\newglossaryentry{Clang} {
    name = {Clang},
    description = {Compilador de código abierto, parte del proyecto LLVM, que fue diseñado para remplazar a GCC},
    see = {LLVM, GCC}
}

\newglossaryentry{Xcode} {
    name = {Xcode},
    description = {IDE de Apple, enfocado en el desarrollo de aplicaciones para sus sistemas operativos}
}

\newglossaryentry{GCC} {
    name = {GCC},
    description = {Compilador de código abierto, desarrollado por el proyecto GNU, que soporta múltiples lenguajes de programación}
}
\newglossaryentry{Seer} {
    name = {Seer},
    description = {Herramienta de análisis de código para programas concurrentes que permite detectar errores de concurrencia que utiliza GDB},
    see = {depurador ,GDB}
}

\newglossaryentrywithacronym{CPU}
{Central Processing Unit}
{Unidad de procesamiento central, componente de un ordenador que interpreta y ejecuta instrucciones}


\newglossaryentrywithacronym{IDE}
{Integrated Development Environment}
{Entorno de desarrollo integrado, un software que combina herramientas para facilitar la programación, como un editor de texto, un compilador, un depurador, entre otros}

\newglossaryentrywithacronym{API}
{Application Programming Interface}
{Conjunto de funciones y procedimientos que permiten la comunicación entre distintos componentes de software}

\newglossaryentrywithacronym{GDB}
{GNU Debugger}
{Depurador de código abierto que permite controlar la ejecución
de programas y analizar su estado en tiempo de ejecución hecho por la Free Software Foundation}

\newglossaryentrywithacronym{LLDB}
{Low Level Debugger}
{Depurador de código abierto que permite controlar la ejecución
de programas y analizar su estado en tiempo de ejecución hecho por el proyecto LLVM}

\newglossaryentrywithacronym{macOS}
{Macintosh Operating System}
{Sistema operativo desarrollado por Apple para sus ordenadores Macintosh}

\newglossaryentrywithacronym{CLI}
{Command Line Interface}
{Tipo de interfaz basada en texto que permite la interacción con un programa a través de comandos escritos en la consola}

\newglossaryentrywithacronym{TUI}
{Text User Interface}
{Tipo de interfaz basada en texto que permite la interacción con un programa a través de elementos visuales, como botones, menús y ventanas}


\makeglossaries
% \makenoidxglossaries  % slower, but no need to do $ makeglossaries report.tex


% usage:
% - \gls{term}: regular stuff
% - \Gls{term}: first uppercase
% - \glspl{term}: plural
% - \Glspl{term}: you know the drill
% - \glsdisp{term}{custom text}: link with custom text
% https://www.overleaf.com/learn/latex/Glossaries

\newglossaryentry{depurar}{
    name = {depurar},
    description = {Proceso de identificar y corregir errores en un programa},
    see = {depurador}
}

\newglossaryentry{software}{
    name = {software},
    description = {Conjunto de programas y datos que permiten el funcionamiento de un sistema informático}
}

\newglossaryentry{depurador} {
    name = {depurador},
    description = {Programa que tiene como objetivo probar y depurar otros programas, a través del control de la ejecución del programa y la inspección del estado del mismo},
    plural = {depuradores},
    see = {proceso}
}

\newglossaryentry{overhead}{
    name = {overhead},
    description = {Tiempo adicional que se necesita para realizar una tarea, que no contribuye directamente al resultado final},
    plural = {overheads}
}

\newglossaryentry{programa concurrente} {
    name = {programa concurrente},
    description = {Programa que utiliza múltiples procesos en ejecución ,que pueden ser ejecutados en paralelo o de forma intercalada, para realizar tareas de forma simultánea},
    plural = {programas concurrentes},
    see = {proceso}
}

\newglossaryentry{registros}{
    name = {registros},
    description = {Memoria de alta velocidad que almacena datos y direcciones de memoria temporales para mejorar el rendimiento de un procesador}
}

\newglossaryentry{concurrencia} {
    name = {concurrencia},
    description = {Propiedad de un sistema que permite la ejecución de múltiples tareas de forma simultánea},
    see = {proceso}
}

\newglossaryentry{lenguaje máquina}{
    name = {lenguaje máquina},
    description = {Lenguaje de programación que representa instrucciones en código binario}
}

\newglossaryentry{ensamblador}{
    name = {ensamblador},
    description = {Lenguaje de programación de bajo nivel que representa instrucciones en código máquina}
}

\newglossaryentry{desensamblador}{
    name = {desensamblador},
    description = {Programa que traduce código máquina a código ensamblador}
}

\newglossaryentry{planificador}{
    name = {planificador},
    description = {Software responsable de gestionar y asignar recursos a los procesos en un sistema operativo para optimizar su ejecución},
    see = {proceso}
}

\newglossaryentry{paralelismo}{
    name = {paralelismo},
    description = {Técnica de programación que consiste en ejecutar múltiples tareas de forma simultánea para mejorar el rendimiento de un programa}
}

\newglossaryentry{cerrojo}{
    name = {cerrojo},
    description = {Mecanismo de sincronización que permite controlar el acceso a recursos compartidos entre múltiples procesos},
    plural = {cerrojos}
}

\newglossaryentry{variable de condición}{
    name = {variable de condición},
    description = {Mecanismo de sincronización que permite a un hilo esperar a que se cumpla una condición antes de continuar su ejecución}
}

\newglossaryentry{condicion de carrera}{
    name = {condicion de carrera},
    description = {Situación en la que el resultado de un programa depende del orden de ejecución de sus instrucciones},
    plural = {condiciones de carrera}
}

\newglossaryentry{interbloqueo}{
    name = {interbloqueo},
    description = {Situación en la que dos o más procesos quedan bloqueados esperando a que se liberen recursos que se están utilizando},
    plural = {interbloqueos}
}

\newglossaryentry{sistemas multicore}{
    name = {sistemas multicore},
    description = {Sistemas que contienen múltiples núcleos de procesamiento en un solo chip, lo que permite ejecutar múltiples tareas de forma simultánea}
}

\newglossaryentry{proceso} {
    name = {proceso},
    description = {Entidad que representa una tarea en ejecución en un sistema operativo},
    plural = {procesos}
}

\newglossaryentry{hilo}{
    name = {hilo},
    description = {Unidad de ejecución más pequeña que un proceso, que comparte recursos con otros hilos de un mismo proceso},
    plural = {hilos}
}

\newglossaryentry{step over} {
    name = {step over},
    description = {Comando de depuración que permite avanzar una instrucción en el código fuente sin entrar en las funciones que se llaman},
    see = {depurar}
}

\newglossaryentry{step into} {
    name = {step into},
    description = {Comando de depuración que permite avanzar una instrucción en el código fuente y entrar en las funciones que se llaman},
    see = {depurar}
}

\newglossaryentry{interfaz gráfica} {
    name = {interfaz gráfica},
    description = {Tipo de interfaz que permite la interacción con un programa a través de elementos visuales, como botones, menús y ventanas}
}

\newglossaryentry{script} {
    name = {script},
    description = {Programas que automatizan tareas repetitivas},
    plural = {scripts}
}

\newglossaryentry{Unix} {
    name = {Unix},
    description = {Familia de sistemas operativos multiusuario y multitarea, desarrollados en los años 70}
}

\newglossaryentry{interfaz máquina-humano} {
    name = {interfaz máquina-humano},
    description = {Conjunto de elementos que permiten la interacción entre un ser humano y una máquina}
}

\newglossaryentry{salida estándar}{
    name = {salida estándar},
    description = {Canal de comunicación que permite la salida de datos de un programa}
}

\newglossaryentry{código abierto}{
    name = {código abierto},
    description = {Software cuyo código fuente es accesible y modificable por cualquier usuario}
}

\newglossaryentry{compilación}{
    name = {compilación},
    description = {Proceso de traducción de un programa escrito en un lenguaje de programación a código máquina}
}

\newglossaryentry{condición de carrera}{
    name = {condición de carrera},
    description = {Situación en la que el resultado de un programa depende del orden de ejecución de sus instrucciones},
    plural = {condiciones de carrera}
}

\newglossaryentry{inyección de código}{
    name = {inyección de código},
    description = {Técnica de programación que consiste en insertar código en un programa en tiempo de ejecución}
}

\newglossaryentry{indeterminista}{
    name = {indeterminista},
    description = {Propiedad de un sistema que no garantiza un resultado predecible},
    see = {determinista},
    plural = {indeterministas}
}

\newglossaryentry{determinista}{
    name = {determinista},
    description = {Propiedad de un sistema que garantiza un resultado predecible},
    see = {indeterminista}
    prural = {deterministas}
}


\newglossaryentry{tiempo de compilación}{
    name = {tiempo de compilación},
    description = {Período durante el cual un programa se traduce a código máquina},
    see = {compilación}
}

\newglossaryentry{tiempo de ejecución}{
    name = {tiempo de ejecución},
    description = {Período durante el cual un programa se ejecuta en un sistema}
}

\newglossaryentry{falso positivo}{
    name = {falso positivo},
    description = {Resultado de una prueba que indica que un error existe cuando en realidad no es así},
    plural = {falsos positivos}
}

\newglossaryentry{barrera}{
    name = {barrera},
    description = {Mecanismo de sincronización que permite a un hilo esperar a que otros hilos alcancen un punto de sincronización antes de continuar su ejecución},
    plural = {barreras}
}

\newglossaryentry{parser}{
    name = {parser},
    description = {Programa que analiza el código fuente de un programa para identificar su estructura y sintaxis}
}

\newglossaryentrywithacronym{CPU}
{Central Processing Unit}
{Unidad de procesamiento central, componente de un ordenador que interpreta y ejecuta instrucciones}


\newglossaryentrywithacronym{IDE}
{Integrated Development Environment}
{Entorno de desarrollo integrado, un software que combina herramientas para facilitar la programación, como un editor de texto, un compilador, un depurador, entre otros}

\newglossaryentrywithacronym{API}
{Application Programming Interface}
{Conjunto de funciones y procedimientos que permiten la comunicación entre distintos componentes de software}


\newglossaryentrywithacronym{CLI}
{Command Line Interface}
{Tipo de interfaz basada en texto que permite la interacción con un programa a través de comandos escritos en la consola}

\newglossaryentrywithacronym{TUI}
{Text User Interface}
{Tipo de interfaz basada en texto que permite la interacción con un programa a través de elementos visuales, como botones, menús y ventanas}

\newglossaryentrywithacronym{GUI}
{Graphical User Interface}
{Tipo de interfaz que permite la interacción con un programa a través de elementos visuales, como botones, menús y ventanas}

